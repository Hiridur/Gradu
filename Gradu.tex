\documentclass[12pt,a4paper,leqno]{report}

%\usepackage[ansinew]{inputenc}
\usepackage[utf8]{inputenc}
\usepackage[T1]{fontenc}
\usepackage[finnish]{babel}
\usepackage{amsthm}
\usepackage{amsfonts}         
\usepackage{amsmath}
\usepackage{amssymb}
\usepackage{enumerate}
\usepackage{enumitem}

\newcommand{\R}{\mathbb{R}}
\newcommand{\C}{\mathbb{C}}
\newcommand{\Q}{\mathbb{Q}}
\newcommand{\N}{\mathbb{N}}
\newcommand{\No}{\mathbb{N}_0}
\newcommand{\Z}{\mathbb{Z}}
\newcommand{\U}{\mathcal{U}}
\newcommand{\T}{\mathcal{T}}
\newcommand{\Pot}{\mathcal{P}}
\newcommand{\diam}{\operatorname{diam}}

\theoremstyle{plain}
\newtheorem{lause}[equation]{Lause}
\newtheorem{lem}[equation]{Lemma}
\newtheorem{prop}[equation]{Propositio}
\newtheorem{kor}[equation]{Korollaari}

\theoremstyle{definition}
\newtheorem{maar}[equation]{Määritelmä}
\newtheorem{konj}[equation]{Konjektuuri}
\newtheorem{esim}[equation]{Esimerkki}

\theoremstyle{remark}
\newtheorem{huom}[equation]{Huomautus}

\pagestyle{plain}
\setcounter{page}{1}
\addtolength{\hoffset}{-1.15cm}
\addtolength{\textwidth}{2.3cm}
\addtolength{\voffset}{0.45cm}
\addtolength{\textheight}{-0.9cm}

\begin{document}

%\maketitle
\begin{titlepage}
  \setlength{\parindent}{0mm}
  \sloppy
  \large \textsc{Helsingin Yliopisto \\
                 Matemaattis-luonnontieteellinen tiedekunta \\
                 Matematiikan ja tilastotieteen laitos}
  \vspace{5mm}
  \hrule height3pt
  \vspace{20mm}
  \begin{center}
    \large Pro gradu -tutkielma
    \linebreak \vfill   
    \huge \textbf{Stone–Čech kompaktisointi}
    \vspace{20mm} \linebreak
    \Large Pekka Keipi \linebreak
    %\normalsize 0000000  % opiskelijanumero
    \vfill
  \end{center}
  \hrule height2pt
  \vspace{15mm}
  Ohjaaja: Erik Elfving
  \hfill
  %18.11.2016    % päivämäärä (esim. 2.2.2002)
  \today
\end{titlepage}

\tableofcontents

\chapter{Johdanto}\label{johd}

%Nicolas Bourbaki is the pseudonym for a group of mathematicians that included Henri Cartan, Claude Chevalley, Jean Dieudonne, and Andres Weil. Mostly French, they emphasized an axiomatic and abstract treatment on all aspects of modern mathematics in Elements de mathematique. The first volume of Elements appeared in 1939. Subsequently, a wide variety of topics have been covered, including works on set theory, algebra, general topology, functions of a real variable, topological vector spaces, and integration. One of the goals of the Bourbaki series is to make the logical structure of mathematical concepts as transparent and intelligible as possible. The books listed below are typical of volumes written in the Bourbaki spirit and now available in English.

\chapter{Esitietoja}
Olkoon $X$ joukko ja $V,W$ sen osajoukkoja.
Merkitään tällöin joukkoilla $V$ ja $W$ seuraavasti: $$V\circ W=\{(x,z)\mid \text{ on olemassa sellainen }y \in X\text{ jolla }(x,y)\in V\text{ ja }(y,z)\in W\}$$ ja $W^2=W\circ W$.
\begin{maar}%Topo2 2.14
Topologian ympäristökanta. 
Olkoon $(X,d)$ topologinen avaruus ja $x\in X$. 
Kokoelma $B(x)$ alkion $x$ ympäristöjä on alkion $x$ ympäristökanta (topologiassa $\T$), jos jokainen alkion $x$ ympäristö sisältää kokoelman $B(x)$ jonkin jäsenen. 
\end{maar}
\begin{esim}
Jos $B$ on avaruuden $(X,\T)$ kanta ja $x\in X$, niin $\{A\mid x\in A\in B\}$ on alkion $x$ eräs ympäristökanta.
Käänteisesti, jos jokaisella $x\in X$ on annettu ympäristökanta $B(x)$ avaruudessa $(X,\T)$, niin $\bigcup\{B(x)\mid x\in X\}$ on avaruuden $(X,\T)$ kanta.
\end{esim}
%\begin{maar}
%Joukon $X$ peite $A$ virittää yksikäsitteisesti
%\end{maar}
\begin{lause}
Olkoon $A$ joukon $X$ peite. Tällöin $A$ on joukon $X$ erään topologian $\T$ esikanta. Lisäksi $\T$ on karkein niistä joukon $X$ topologioista, joilla $A\in\T$. Tämä topologia $\T$ on peitteen $A$ yksikäsitteisesti määräämä, ja sitä sanotaan peitteen $A$ virittämäksi joukon $X$ topologiaksi.
\begin{proof}
%topo2 2.19
\end{proof}
\end{lause}
\chapter{Uniformiset rakenteet}
Tässä kappaleessa tutustutaan uniformisiin rakenteisiin ja näiden keskeisiin ominaisuuksiin \cite{Eom1}.
%\\
%\\
%Merkitään joukkoilla $V$ ja $W$ seuraavasti: $$V\circ W=\{(x,z)\mid \text{ on olemassa sellainen }y \in X\text{ jolla }(x,y)\in V\text{ ja }(y,z)\in W\}$$ ja $W^2=W\circ W$.
\begin{maar}\label{uniformi_maar}
Uniforminen rakenne (tai uniformisuus) joukolle $X$ annetaan karteesisen tulon $X\times X$ potenssijoukon $\Pot(X\times X)$ osajoukkona $\U$, jolle pätee %kaikilla $V\in \U$
\begin{enumerate} [label=(U\arabic*),ref=(U\arabic*)]
\item\label{F_I} Jos $V\in \U$ ja $V\subset W\subset X\times X$ niin $ W\in\U$,
\item\label{F_II} Jokainen äärellinen leikkaus joukon $\U$ alkioista kuuluu joukkoon $\U$,
\item\label{U_I} Joukko $\{(x,x)\mid x\in X\}$ on jokaisen joukon $V\in\U$ osajoukko,
\item\label{U_II} Jos $V\in\U$, niin $V^{-1}=\{(y,x)\mid (x,y)\in V\}\in\U$,
\item\label{U_III} Jos $V\in \U$, niin on olemassa sellainen $W\in \U$, jolla $ W^2\subset V$.%$ W\circ W\subset V$.
\end{enumerate}
Uniformisen rakenteen muodostavia joukkoja $ V\in\U$ sanotaan uniformisuuden $\U$ lähistöksi. Joukkoa $X$ joka on varustettu uniformisuudella $\U$ sanotaan univormiseksi avaruudeksi.
\end{maar}
\begin{huom}
Uniformisuuden $\U$ lähistöön (entourage) $V\in\U $ kuuluvan pisteparin $(x,y)\in V$ pisteiden $x,y\in X$ sanotaan olevan $V-$lähellä, tarpeeksi lähellä tai mielivaltaisen lähellä toisiaan.
\end{huom}
\begin{huom}
%Mikäli muut ehdot pätevät, voidaan ehdot (U$\ref{U_II}$) ja (U$\ref{U_III}$) korvata yhtäpitävällä ehdolla 
Mikäli muut ehdot pätevät, voidaan ehdot \ref{U_II} ja \ref{U_III} korvata yhtäpitävällä ehdolla 
\begin{enumerate}
%\item[(U$\ref{U_II}$a)] \label{Uaehto} 
\item[(Ua)] \label{Uaehto} 
Jos $V\in \U$, niin on olemassa sellainen $W\in \U$, jolla $ W\circ W^{-1}\subset V$.
\end{enumerate}
%Ehto (U$\ref{U_II}$\ref{Uaehto}) seuraa
%ehdoista (U$\ref{U_II}$) ja (U$\ref{U_III}$), sillä\dots
%
%Ehdot 
\end{huom}
\begin{huom}
%Jos joukko $X$ on tyhjä, niin ehdon (U$\ref{U_I}$) 
Jos joukko $X$ on tyhjä, niin ehdon \ref{U_I} 
nojalla joukon $X$ uniformiteetti $\U$ on tyhjä. Erityisesti $\{\emptyset\}$ on joukon $X$ ainoa ehdot täyttävä uniformiteetti, jos joukko $X$ on tyhjä.
\end{huom}
\begin{maar}
Olkoon $X$ joukko ja joukko $\U\subset X\times X$ sen uniformiteetti. Tällöin lähistöjen joukko $B\subset \U$ on uniformiteetin $\U$ kanta, jos jokaiselle lähistölle $V\in\U$ löytyy kannan alkio $W\in B$, jolla pätee $W\subset V$.
\end{maar}
\begin{maar}
Olkoon $X$ joukko. Joukko $B\subset \Pot (X\times X)$ on joukon $X$ uniformisuuden kanta, jos joukolle $B$ pätee
%Olkoon $X$ joukko ja $B\subset \Pot (X\times X)$ sellainen joukko, jolla pätee
\begin{enumerate} [label=(B\arabic*)]
\item\label{B_I} Jos $V_1,V_2\in B$ niin on olemassa sellainen $V_3\in B$, jolla $V_3\subset V_1\cap V_2$,
\item\label{U'_I} Joukko $\{(x,x)\mid x\in X\}$ on jokaisen joukon $V\in B$ osajoukko,
\item\label{U'_II} Jos $V\in B$, niin on olemassa sellainen $V'\in B$, jolla $V'\subset V^{-1}%=\{(y,x)\mid (x,y)\in V\}
$,
\item\label{U'_III} Jos $V\in B$, niin on olemassa sellainen $W\in B$, jolla $ W^2\subset V$.%$ W\circ W\subset V$.
\end{enumerate}
\end{maar}
\begin{lause}
%\emph{Uniformisen avaruuden topologia.}
Uniformisen avaruuden topologia.
Olkoon joukko $X$ varustettu uniformisuudella $\U$.
Olkoon $x\in X$ alkio ja $V\in\U$ lähistö avaruudessa $X$. Olkoon %$V(x)\subset X$,
\begin{equation*}V(x)=\{ y\in X\mid (x,y)\in V \}
\text{ ja }
B(x)=\{ V(x)\mid V\in\U \}
\end{equation*}
joukkoja.
Uniformiteetti $\U$ määrää topologian joukolle $X$ niin, että joukko $V(x)$ on (lähistön $V$ määräämä) ympäristö alkiolle $x$ ja joukko $B(x)$ on alkion $x$ kaikkien ympäristöjen joukko kyseisessä topologiassa.
\begin{proof}
Olkoon joukko $X$ varustettu uniformisuudella $\U$.
Olkoon $x\in X$ alkio, $V\in\U$ lähistö avaruudessa $X$ ja joukot $V(x)$ ja $B(x)$ kuten edellä. 
%
Alkiolle $x$ pätee $x\in V(x)$, joten joukko $V(x)$ on epätyhjä. 
Jokaiselle lähistölle $V,W\in\U$ pätee 
\begin{align*}
V(x)\cup W(x) =&\{ y\in X\mid (x,y)\in V \text{ tai } (x,y)\in W \}\\
=&\{ y\in X\mid (x,y)\in V\cup W \}
\in B(x),
\end{align*} 
ja 
\begin{align*}
V(x)\cap W(x) =&\{ y\in X\mid (x,y)\in V \text{ ja } (x,y)\in W \}\\
=&\{ y\in X\mid (x,y)\in V\cap W \}
\in B(x)
\end{align*} 
sillä määritelmän \ref{uniformi_maar} ehtojen \ref{F_I} ja \ref{F_II} nojalla $ V\cup W \in B$ ja $ V\cap W \in B$. Tällöin on olemassa yksikäsitteinen topologia, jossa $B(x)$ on alkion $x$ kaikkien ympäristöjen joukko.\cite{Eom1}
%yhdisteen kohdassa tarvitaan yleisempi tapaus. ehto V_4 puuttuu sivu19 Eom1
\end{proof}
\end{lause}
\begin{huom}
Uniformiteetin määräämää topologiaa sanotaan uniformiteetin indusoimaksi topologiaksi.
\end{huom}
\begin{maar}
%Uniformisesti jatkuvat kuvaukset. 
Olkoon $X$ ja $X'$ uniformeja avaruuksia 
ja $f\colon X\rightarrow X'$ kuvaus.
Kuvaus $f$ on \emph{uniformisti jatkuva}, jos jokaiselle avaruuden $X'$ lähistölle $V'$ on olemassa avaruuden $X$ lähistö $V$ niin, että jokaiselle $(x,y)\in V$ pätee $(f(x),f(y))\in V'$. 
%Toisin sanoen jokaiselle avaruuden $X'$ lähistölle $V'$ on olemassa.
\end{maar}
\begin{kor}\label{uniformi alkukuva}
Olkoon $X$ ja $X'$ uniformeja avaruuksia 
ja $f\colon X\rightarrow X'$ uniformisti jatkuva kuvaus. 
Olkoon $g\colon X\times X\rightarrow X'\times X'$ %sellainen 
kuvaus, jolla pätee $g(x,y)=(f(x),f(y))$ kaikilla $x,y\in X$. 
Tällöin jos $V'$ on avaruuden $X'$ lähistö, niin alkukuva $g^{\leftarrow}V'$ on avaruuden $X$ lähistö.
\end{kor}
\begin{lause}
Jokainen uniformisti jatkuva kuvaus on jatkuva uniformien määräämien topologioiden suhteen.
\begin{proof}
Olkoon $X$ ja $X'$ uniformeja avaruuksia 
ja $f\colon X\rightarrow X'$ uniformisti jatkuva kuvaus. 
Olkoon $g\colon X\times X\rightarrow X'\times X'$ %sellainen 
kuvaus, jolla pätee $g(x,y)=(f(x),f(y))$ kaikilla $x,y\in X$. 
Olkoon $V'$ avaruuden $X'$ lähistö ja $x'\in X'$ alkio. 
Korollaarin \ref{uniformi alkukuva} mukaan lähistön $V'$ alkukuva $g^{\leftarrow}V'$ on avaruuden $X$ lähistö. 
Avaruuden $X'$ uniformiteetin määrittämässä topologiassa kaava $V'(x')$ määrää alkion $x'$ ympäristön.
Tällöin koska $g^{\leftarrow}V'$ on avaruuden $X$ lähistö, niin $g^{\leftarrow}V'(g^{\leftarrow}(x'))$ on alkion $x'$ alkukuvan $g^{\leftarrow}(x')\in X$ ympäristö. 
%erikseen 1-käs? eli alkukuva ympäristöstä lähistön sijaan?

Siis ympäristön alkukuva on ympäristö ja siten uniformisti jatkuva kuvaus on jatkuva.
\end{proof}
\end{lause}
\begin{lause}
Olkoon $X$, $X'$ ja $X''$ uniformeja avaruuksia 
ja $f\colon X\rightarrow X'$ ja $g\colon X'\rightarrow X''$ uniformisti jatkuvia kuvaksia. Tällöin yhdistetty kuvaus $g\circ f\colon X\rightarrow X''$ on uniformisti jatkuva.
\begin{proof}
Lisätään myöhemmin.
\end{proof}
\end{lause}
\begin{maar}
Olkoon $X$ ja $X'$ uniformeja avaruuksia 
ja $f\colon X\rightarrow X'$ bijektiivinen kuvaus. Kuvaus $f$ on isomorfia, jos sekä kuvaus $f$ että sen käänteiskuvaus $f^{-1}$ ovat uniformisti jatkuvia.
\end{maar}
\begin{maar}
\emph{Uniformiteettien vertailtavuus.} 
Olkoon $X$ joukko ja $\U_1$ ja $\U_2$ uniformiteetteja joukolle $X$. 
Uniformiteetti $\U_1$ on hienompi kuin uniformiteetti $\U_2$, 
jos identtinen kuvaus $id\colon (X,\U_1)\rightarrow (X,\U_2)$ on uniformisti jatkuva. Tällöin uniformiteetti $\U_2$ on karkeampi kuin uniformiteetti $\U_1$. 
Jos lisäksi pätee $\U_1\neq\U_2$, niin $\U_1$ on aidosti hienompi kuin $\U_2$ ja vastaavasti $\U_2$ on aidosti karkeampi kuin $\U_1$. 
Kahta uniformiteettia $\U_1$ ja $\U_2$ on mahdollista vertailla, jos $\U_1$ on hienompi kuin $\U_2$.
\end{maar}
\begin{kor}
Olkoon $X$ joukko ja $\U_1$ ja $\U_2$ uniformiteetteja joukolle $X$. 
Uniformiteetti $\U_1$ on hienompi kuin uniformiteetti $\U_2$ jos ja vain jos jokaisella lähistöllä $V\in\U_2$ pätee $V\in\U_1$.
\end{kor}
\begin{kor}
Olkoon $X$ joukko, $\U_1$ ja $\U_2$ uniformiteetteja joukolle $X$ ja 
$\U_1$ on hienompi kuin $\U_2$. Tällöin uniformiteetin $\U_1$ indusoima topologia on hienompi kuin uniformiteetin $\U_2$ indusoima topologia.
\begin{proof}
Ympäristökannat.
\end{proof}
\end{kor}
\begin{maar}\label{kuvausperheen indusoima}
\emph{Initial uniformities. Kuvausperheen indusoima uniformiteetti.} 
Olkoon $X$ joukko ja $Y_i$ uniformiteetilla varustettuja joukkoja kaikilla $i\in I$, jollain indeksijoukolla $I$. 
%Olkoon $\U_i$ joukon $Y_i$ uniformiteetti kaikilla $i\in I$.
Olkoon $f_i\colon X\rightarrow Y_i$ uniformisti jatkuvia kuvauksia kaikilla $i\in I$. 
Olkoon $g=f_i\times f_i\colon X\times X\rightarrow Y_i\times Y_i$ %sellainen 
%kuvaus, jolla pätee $g(x,y)=(f_i(x),f_i(y))$ kaikilla $x,y\in X$. 
kuvaus kaikilla $i\in I$.
Olkoon $$B=\left\{\bigcap_{i\in I}g^{\leftarrow}(V_i)\mid V_i \in\U_i \right\}$$
missä $\U_i$ on avaruuden $Y_i$ uniformiteetti.
Tällöin $B$ on kanta eräälle avaruuden $X$ uniformiteetille $\U$. 
Kyseinen uniformiteetti $\U$ on karkein niistä uniformiteeteista, joiden suhteen kaikki kuvaukset $f_i$ ovat uniformisti jatkuvia.
\end{maar}
\begin{maar}
\emph{Uniformiteettien pienin yläraja}
%Olkoon $X$ joukko%, $I$ indeksijoukko
% ja jokaisella $i\in I$ olkoon $\U_i$ uniformiteetti joukolle $X$. 
Olkoon $X$ joukko ja $I$ jokin indeksijoukko.
Olkoon $(\U_i)_{i\in I}$ perhe uniformiteetteja joukolle $X$.
Tällöin perheen $(\U_i)_{i\in I}$ \emph{pienin yläraja} on uniformiteetti $\U$, joka on kuvausten $id\colon X\rightarrow (X,\U_i)$ määritelmän \ref{kuvausperheen indusoima} mukaisesti indusoima.
\end{maar}
\chapter{Pseudometriikat}
Tässä kappaleessa tutustutaan pseudometriikoihin, jotka ovat tavanomaisten metriikoiden yleistys. Lisää aiheesta \cite{Eom2}.
%\\
\begin{maar}
Olkoon $X$ joukko ja $f\colon X\times X\rightarrow [0,+\infty]$ 
% sellainen kuvaus, joka täyttää seuraavat ehdot:
kuvaus. Kuvaus $f$ on pseudometriikka, jos seuraavat ehdot pätevät:
\begin{enumerate} [label=(P\arabic*),ref=(P\arabic*)]
\item\label{EC_I} $f(x,x)=0$ kaikilla $x\in X$,
\item\label{EC_II} $f(x,y)=f(y,x)$ kaikilla $x,y\in X$,
\item\label{EC_III} $f(x,y)\leq f(x,z)+f(z,y)$ kaikilla $x,y,z\in X$.
\end{enumerate}
\end{maar}
\begin{huom}
%Metriikka %$m\colon X\times X\rightarrow 
%on sellainen pseudometriikka, jolta lisäksi vaaditaan, että joukko $X$ on epätyhjä ja että 
Pseudometriikasta saadaan metriikka, jos rajoitutaan äärellisiin arvoihin ja vahvistetaan ehtoa \ref{EC_I} muotoon
\begin{enumerate} [label=(M\arabic*),ref=(M\arabic*)]
\item\label{M1} $f(x,y)=0\Leftrightarrow x=y$ kaikilla $x,y\in X.$
\end{enumerate}
\end{huom}
\begin{esim}%1)
Euklidinen etäisyys on pseudometriikka.
\end{esim}
\begin{esim}%2)
Olkoon $X$ epätyhjä joukko ja $f\colon X\times X\rightarrow [0,+\infty]$ sellainen kuvaus, jolla
\begin{equation*}
f(x,y) = \begin{cases} 0, & \mbox{jos } x=y\\
\infty, & \mbox{muulloin. } \end{cases}
\end{equation*}
Tällöin $f$ on pseudometriikka.
\end{esim}
\begin{esim}%3)
Olkoon $X$ epätyhjä joukko ja $g\colon X\rightarrow \R$ (äärellisarvoinen) kuvaus. Tällöin kuvaus $f\colon X\times X\rightarrow[0,+\infty]$ kaavalla $f(x,y)=|g(x)-g(y)|$ on pseudometriikka.
\end{esim}
\begin{esim}%4)
Olkoon $X$ kaikkien muotoa $g\colon [0,1]\rightarrow \R$ olevien jatkuvien kuvausten joukko. % väliltä $[0,1]$ reaaliluvuille. 
Tällöin kuvaus $f\colon X\times X\rightarrow [0,+\infty]$ kaavalla $f(x,y)=\int_0^1 |x(t)-y(t)|dt$ määrittää pseudometriikan joukolle $X$.
\end{esim}
\begin{huom}
%Metriikoista tutut ominaisuudet, kuten
Ehdosta \ref{EC_III} seuraa, että jos $f(x,z)+f(z,y)<\infty$ niin $f(x,y)<\infty$. Tällöin koska kaavat $f(x,z)\leq f(x,y)+f(y,z)$ ja $f(y,z)\leq f(y,x)+f(x,z)$ pätevät, niin myös kaava $|f(x,z)-f(z,y)|\leq f(x,y)$ pätee.
\end{huom}
\begin{esim}%1
%Olkoon $f\colon X\times X\rightarrow$ pseudometriikka
Olkoon $f$ pseudometriikka. Tällöin myös $\lambda f$ on pseudometriikka, jos kaavat $(\lambda f) (x)=\lambda (f (x))$ ja $0<\lambda <+\infty$ pätevät.
\end{esim}
\begin{esim}%2
Olkoon $(f_i)_{i\in I}$ perhe joukon $X$ pseudometriikoita. 
Tällöin summakuvaus 
\begin{equation*}
f\colon X\times X\rightarrow [0,+\infty],
%\text{ kaavalla } 
f(x,y)=\sum_{i\in I}f_i(x,y)%\text{ kaikilla }x,y\in X
\end{equation*} 
%$f\colon X\times X\rightarrow [0,+\infty]$ kaavalla $ f(x,y)=\sum_{i\in I}f_i(x,y)$ 
kaikilla $x,y\in X$ 
on pseudometriikka.
\end{esim}
\begin{esim}%3
Olkoon $(f_i)_{i\in I}$ perhe joukon $X$ pseudometriikoita. 
Tällöin kaikilla alkioilla $x,y\in X$ kaavasta $$f_i(x,y)\leq f_i(x,z)+f_i(z,y)$$ seuraa kaava $$\sup_{i\in I}f_i(x,y)\leq \sup_{i\in I}\left(f_i(x,z)+f_i(z,y)\right).$$
%\begin{align*}
%&f_i(x,y)\leq f_i(x,z)+f_i(z,y)\\
%\Rightarrow &\sup_{i\in I}f_i(x,y)\leq \sup_{i\in I}\left(f_i(x,z)+f_i(z,y)\right)
%\end{align*} 
Tällöin kuvaus 
\begin{equation*}
f\colon X\times X\rightarrow [0,+\infty],
%\text{ kaavalla } 
f(x,y)=\sup_{i\in I}f_i(x,y)
\end{equation*} 
kaikilla $x,y\in X$ on pseudometriikka. 
\end{esim}
\chapter{Pseudometriikan määrittämä uniformisuus}\label{pseudo_uniformi}
Oletamme koko luvun \ref{pseudo_uniformi} ajan, että $\R_+=\{a\in\R\mid a>0\}$.
\begin{maar}
Olkoon $a\in\R_+$ reaaliluku ja $ \U_a\in\Pot( \R^n\times\R^n)$ joukko kaavalla
$$\U_a=\{ (x,y)\mid |x-y|\leq a,x,y\in\R^n\}$$
Nyt $B=\{\U_a\mid a\in\R_+\}$ muodostaa uniformisuuden kannan avaruudelle $\R^n$:
\begin{enumerate} [label=(B\arabic*)]
\item %\ref{B_I} 
%Jos $V_1,V_2\in B$ niin on olemassa sellainen $V_3\in B$, jolla $V_3\subset V_1\cap V_2$,
\item%\ref{U'_I} 
%Joukko $\{(x,x)\mid x\in X\}$ on jokaisen joukon $V\in B$ osajoukko,
\item%\ref{U'_II} 
%Jos $V\in B$, niin on olemassa sellainen $V'\in B$, jolla $V'\subset V^{-1}%=\{(y,x)\mid (x,y)\in V\}$,
\item%\ref{U'_III} 
%Jos $V\in B$, niin on olemassa sellainen $W\in B$, jolla $ W^2\subset V$.%$ W\circ W\subset V$.
\end{enumerate}
\end{maar}

\begin{huom}
\end{huom}
\begin{lause}
\end{lause}
\begin{maar}
\end{maar}
\begin{kor}
\end{kor}
\begin{esim}
\end{esim}
\begin{lem}
\end{lem}
\begin{thebibliography}{9}

\bibitem{Eom1}
Nicolas Bourbaki: General Topology Part 1, 1.\ painos, Hermann, 1966.

\bibitem{Eom2}
Nicolas Bourbaki: General Topology Part 2, 1.\ painos, Hermann, 1966.

%\bibitem{Bor}
%Karol Borsuk: Theory of retracts, %n.\ painos, 
%Państwowe Wydawn. Naukowe, 1967.
%
%\bibitem{Topo1}
%Jussi Väisälä: Topologia I, 4.\ korjattu painos, Limes ry, 2007.
%
\bibitem{Topo2}
Jussi Väisälä: Topologia II, 2.\ korjattu painos, Limes ry, 2005.

%\bibitem{Hei}
%Juha Heinonen: Geometric embeddings of metric spaces, luentomoniste, Jyväskylän yliopisto, 2003
%Sheldon (not really) Ross: A First Course in Probability, 5th edition, Prentice-Hall, 1998.

%\bibitem{Tuo}
%Pekka (not really) Tuominen: Todennäköisyyslaskenta I, 5.\ painos, Limes ry, 2000.

\end{thebibliography}

\end{document}
