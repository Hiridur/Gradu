\documentclass[12pt,a4paper,leqno]{report}

%\usepackage[ansinew]{inputenc}
\usepackage[utf8]{inputenc}
\usepackage[T1]{fontenc}
\usepackage[finnish]{babel}
\usepackage{amsthm}
\usepackage{amsfonts}         
\usepackage{amsmath}
\usepackage{amssymb}
\usepackage{enumerate}
\usepackage{enumitem}

\newcommand{\R}{\mathbb{R}}
\newcommand{\C}{\mathbb{C}}
\newcommand{\Q}{\mathbb{Q}}
\newcommand{\N}{\mathbb{N}}
\newcommand{\No}{\mathbb{N}_0}
\newcommand{\Z}{\mathbb{Z}}
\newcommand{\U}{\,\mathcal{U}}
\newcommand{\T}{\mathcal{T}}
\newcommand{\Pot}{\mathcal{P}}
\newcommand{\F}{\mathcal{F}}
\newcommand{\B}{\mathcal{B}}
\newcommand{\diam}{\operatorname{diam}}

\theoremstyle{plain}
\newtheorem{lause}[equation]{Lause}
\newtheorem{lem}[equation]{Lemma}
\newtheorem{prop}[equation]{Propositio}
\newtheorem{kor}[equation]{Korollaari}

\theoremstyle{definition}
\newtheorem{maar}[equation]{Määritelmä}
\newtheorem{konj}[equation]{Konjektuuri}
\newtheorem{esim}[equation]{Esimerkki}

\theoremstyle{remark}
\newtheorem{huom}[equation]{Huomautus}

\pagestyle{plain}
\setcounter{page}{1}
\addtolength{\hoffset}{-1.15cm}
\addtolength{\textwidth}{2.3cm}
\addtolength{\voffset}{0.45cm}
\addtolength{\textheight}{-0.9cm}

\begin{document}

%\maketitle
\begin{titlepage}
  \setlength{\parindent}{0mm}
  \sloppy
  \large \textsc{Helsingin Yliopisto \\
                 Matemaattis-luonnontieteellinen tiedekunta \\
                 Matematiikan ja tilastotieteen laitos}
  \vspace{5mm}
  \hrule height3pt
  \vspace{20mm}
  \begin{center}
    \large Pro gradu -tutkielma
    \linebreak \vfill   
    \huge \textbf{Stone–Čech kompaktisointi}
    \vspace{20mm} \linebreak
    \Large Pekka Keipi \linebreak
    %\normalsize 0000000  % opiskelijanumero
    \vfill
  \end{center}
  \hrule height2pt
  \vspace{15mm}
  Ohjaaja: Erik Elfving
  \hfill
  %18.11.2016    % päivämäärä (esim. 2.2.2002)
  \today
\end{titlepage}

\tableofcontents

\chapter{Johdanto}\label{johd}

%Nicolas Bourbaki is the pseudonym for a group of mathematicians that included Henri Cartan, Claude Chevalley, Jean Dieudonne, and Andres Weil. Mostly French, they emphasized an axiomatic and abstract treatment on all aspects of modern mathematics in Elements de mathematique. The first volume of Elements appeared in 1939. Subsequently, a wide variety of topics have been covered, including works on set theory, algebra, general topology, functions of a real variable, topological vector spaces, and integration. One of the goals of the Bourbaki series is to make the logical structure of mathematical concepts as transparent and intelligible as possible. The books listed below are typical of volumes written in the Bourbaki spirit and now available in English.

\chapter{Esitietoja}
Käytämme koko tutkielman ajan merkintää $\R_+=\{a\in\R\mid a>0\}$.
\\
\\
\begin{maar}
Olkoon $X$ joukko ja $V$ ja $W$ karteesisen tulon $X\times X$ osajoukkoja.
Merkitään tällöin %joukkoilla $V$ ja $W$ 
seuraavasti: 
$$V\circ W=\{(x,z)\mid \text{on olemassa sellainen }y \in X,\text{ jolla }(x,y)\in V\text{ ja }(y,z)\in W\},$$ 
%$V\circ W=\{(x,z)\mid$ on olemassa sellainen $y \in X$, jolla $(x,y)\in V$ ja $(y,z)\in W\},$ 
$W^2=W\circ W$ ja $W^n=W\circ W^{n-1}$.
\end{maar}
\begin{maar}
Topologisen avaruuden $(X,\T)$ osajoukko $A\subset X$ on avoin, 
jos se on kokoelman $\T\subset\Pot(X)$ alkio.
\end{maar}
\begin{maar}%Topo2 ?
%\emph{Ympäristö.} 
%Olkoon $(X,\T)$ topologinen avaruus ja $x\in X$ alkio. 
%Osa\-jouk\-ko $A\subset X$, johon alkio $x$ kuuluu, on alkion $x$ \emph{ympäristö}, 
%jos on olemassa avoin osajoukko $B\subset X $, joka sisältää osajoukon $A$.
Topologisen avaruuden $(X,\T)$ 
osa\-jouk\-ko $A\subset X$ on alkion $x$ \emph{ympäristö}, 
jos on olemassa sellainen avoin osajoukko $B\subset X $, jolla $x\in B\subset A$.
\end{maar}
\begin{huom}
Topologisen avaruuden $(X,\T)$ avoin osajoukko $A\subset X$ on 
jokaisen alkionsa $x\in A$ ympäristö.
\end{huom}
\begin{maar}%Topo2 2.14
\emph{Ympäristökanta.} 
Olkoon $(X,\T)$ topologinen avaruus ja $x\in X$ alkio. 
Kokoelma $B(x)$ alkion $x$ ympäristöjä on alkion $x$ \emph{ympäristökanta} 
(topologiassa $\T$), jos jokainen alkion $x$ ympäristö sisältää 
kokoelman $B(x)$ jonkin jäsenen. 
\end{maar}
%\begin{maar}
%Olkoon $(X,\T)$ topologinen avaruus, $x\in X$ alkio ja . 
%Alkion $x$ kaikkien ympäristöjen kokoelmalle pätee seuraavat ehdot: 
%\begin{enumerate} [label=(V\arabic*),ref=(V\arabic*)]
%\item Jos 
%\end{enumerate} 
%\end{maar}
\begin{kor}\label{kaikki_ystöt}
Topologisen avaruuden $(X,\T)$ alkion $x\in X$ kaikkien ympäristöjen kokoelma $B(x)$ on alkion $x$ ympäristökanta.
\end{kor}
\begin{esim}
Jos joukkoperhe $B\subset\Pot(X)$ on avaruuden $(X,\T)$ kanta ja 
$x\in X$ alkio, niin joukko 
$B(x)=\{A\mid x\in A\in B\}$ on alkion $x$ eräs ympäristökanta.
Käänteisesti, jos jokaisella alkiolla $x\in X$ on annettu ympäristökanta 
$B(x)$ avaruudessa $(X,\T)$, niin kokoelma $\bigcup\{B(x)\mid x\in X\}$ 
on avaruuden $(X,\T)$ kanta.
\end{esim}
%\begin{maar}
%Joukon $X$ peite $A$ virittää yksikäsitteisesti
%\end{maar}
\begin{lause}
Olkoon $A$ joukon $X$ peite. Tällöin $A$ on joukon $X$ erään topologian $\T$ esikanta. 
Lisäksi $\T$ on karkein niistä joukon $X$ topologioista, joilla $A\subset\T$. 
Tämä topologia $\T$ on peitteen $A$ yksikäsitteisesti määräämä, ja sitä sanotaan peitteen $A$ virittämäksi joukon $X$ topologiaksi.
\begin{proof}
Topologia II \cite{Topo2} lause 2.19.
%topo2 2.19
\end{proof}
\end{lause}
\begin{maar}
\emph{Topologioiden vertailu.}
Olkoon $X$ joukko ja $\T_1$ ja $\T_2$ topologioita joukolle $X$.
Topologia $\T_2$ on karkeampi kuin topologia $\T_1$, 
jos jokaisella $ U\in\T_2$ pätee $ U\in\T_1$, eli $ \T_2\subset\T_1$. 
Tällöin $\T_1$ on hienompi kuin $\T_2$.
\end{maar}
\chapter{Uniformiset rakenteet}
Tässä kappaleessa tutustutaan uniformisiin rakenteisiin ja näiden keskeisiin ominaisuuksiin \cite{Eom1}.
%\\
%\\
%Merkitään joukkoilla $V$ ja $W$ seuraavasti: $$V\circ W=\{(x,z)\mid \text{ on olemassa sellainen }y \in X\text{ jolla }(x,y)\in V\text{ ja }(y,z)\in W\}$$ ja $W^2=W\circ W$.
\begin{maar}\label{uniformi_maar}
Uniforminen rakenne (tai uniformiteetti) joukolle $X$ annetaan karteesisen tulon $X\times X$ potenssijoukon $\Pot(X\times X)$ osajoukkona $\U$, jolle pätee %kaikilla $V\in \U$
\begin{enumerate} [label=(U\arabic*),ref=(U\arabic*)]
\item\label{F_I} Jos $V\in \U$ ja $V\subset W\subset X\times X$ niin $ W\in\U$,
\item\label{F_II} Jokainen äärellinen leikkaus joukon $\U$ alkioista kuuluu joukkoon $\U$,
\item\label{U_I} Joukko $\{(x,x)\mid x\in X\}$ on jokaisen joukon $V\in\U$ osajoukko,
\item\label{U_II} Jos $V\in\U$, niin $V^{-1}=\{(y,x)\mid (x,y)\in V\}\in\U$,
\item\label{U_III} Jos $V\in \U$, niin on olemassa sellainen $W\in \U$, jolla $ W^2\subset V$.%$ W\circ W\subset V$.
\end{enumerate}
Uniformisen rakenteen muodostavia joukkoja $ V\in\U$ sanotaan uniformiteetin $\U$ lähistöiksi. 
Uniformiteetilla $\U$ varustettua joukkoa $X$ sanotaan uniformiseksi avaruudeksi.
\end{maar}
\begin{huom}
Uniformiteetin $\U$ lähistöön (entourage) $V\in\U $ kuuluvan 
pisteparin $(x,y)\in V$ pisteiden $x,y\in X$ sanotaan olevan $V-$lähellä, 
tarpeeksi lähellä tai mielivaltaisen lähellä toisiaan.
\end{huom}
\begin{huom}
%Mikäli muut ehdot pätevät, voidaan ehdot (U$\ref{U_II}$) ja (U$\ref{U_III}$) korvata yhtäpitävällä ehdolla 
Mikäli muut ehdot pätevät voidaan ehdot \ref{U_II} ja \ref{U_III} korvata yhtäpitävällä ehdolla: 
\begin{enumerate} [label=(Ua),ref=(Ua)]
%\item[(U$\ref{U_II}$a)] \label{Uaehto} 
\item \label{Uaehto} 
%\item[(Ua)] \label{Uaehto} 
Jos $V\in \U$, niin on olemassa sellainen $W\in \U$, jolla $ W\circ W^{-1}\subset V$.
\end{enumerate}
%Ehto (U$\ref{U_II}$\ref{Uaehto}) seuraa
%ehdoista (U$\ref{U_II}$) ja (U$\ref{U_III}$), sillä\dots
%
%Ehdot 
\end{huom}
\begin{huom}
%Jos joukko $X$ on tyhjä, niin ehdon (U$\ref{U_I}$) 
%Jos joukko $X$ on tyhjä, niin ehdon \ref{U_I} 
%nojalla joukon $X$ uniformiteetti $\U$ on tyhjä. Erityisesti kokoelma $\{\emptyset\}$ on joukon $X$ ainoa ehdot täyttävä uniformiteetti, jos joukko $X$ on tyhjä.
Kokoelma $\{\emptyset\}$ on ainoa ehdot täyttävä uniformiteetti tyhjälle joukolle.
\end{huom}
\begin{maar}\label{uniformi_kanta}
Olkoon $X$ joukko ja kokoelma $\U\subset \Pot (X\times X)$ uniformiteetti 
joukolle $X$. Tällöin lähistöjen joukko $B\subset \U$ on 
uniformiteetin $\U$ \emph{kanta}, jos jokaiselle lähistölle $V\in\U$ löytyy 
kannan alkio $W\in B$, jolla pätee $W\subset V$.
\end{maar}
%\begin{kor}
%Uniformiteetin kannas
%\end{kor}
\begin{lause}\label{uniformin kannan maar}
Olkoon $X$ joukko. 
%Olkoon joukko $X$ varustettu uniformiteetilla $\U$. 
%Kokoelma $B\subset \U$ on uniformiteetin $\U$ kanta, 
Kokoelma $B\subset \Pot (X\times X)$ on joukon $X$ erään uniformiteetin kanta, 
jos kokoelmalle $B$ pätee
%Olkoon $X$ joukko ja $B\subset \Pot (X\times X)$ sellainen joukko, jolla pätee
\begin{enumerate} [label=(B\arabic*)]
\item\label{B_I} Jos $V_1,V_2\in B$ niin on olemassa sellainen $V_3\in B$, jolla $V_3\subset V_1\cap V_2$,
\item\label{U'_I} Joukko $\{(x,x)\mid x\in X\}$ on jokaisen joukon $V\in B$ osajoukko,
\item\label{U'_II} Jos $V\in B$, niin on olemassa sellainen $V'\in B$, jolla $V'\subset V^{-1}%=\{(y,x)\mid (x,y)\in V\}
$,
\item\label{U'_III} Jos $V\in B$, niin on olemassa sellainen $W\in B$, jolla $ W^2\subset V$.%$ W\circ W\subset V$.
\end{enumerate}
\begin{proof}
%Olkoon joukko $X$ varustettu uniformiteetilla $\U$. 
Olkoon $X$ joukko ja $B\subset \Pot (X\times X)$ sellainen kokoelma, jolle pätevät ehdot \ref{B_I}-\ref{U'_III}. %karteesisen tulon $X\times X$ osajoukkoja. 
Tällöin olkoon 
$$\U_B=\{W\subset X\times X\mid V\subset W\text{ jollain } V\in B\}$$ 
%$\U_B\subset\Pot (X\times X)$ sellainen kokoelma, 
%jonka alkioina ovat kaikki sellaiset joukot $V'\subset$
%jolla pätee seuraava ehto: Jos $V\in B$ ja $V\subset W\subset X\times X$ niin $ W\in\U_B$. 
Joukon määrittelystä seuraa $ B\subset\U_B$ ja että jokaiselle lähistölle $W\in\U$ löytyy 
kannan alkio $V\in B$, jolla pätee $V\subset W$. 
Näin ollen riittää tarkistaa, että $\U_B$ on uniformiteetti. 
%Tällöin olkoon 
%%$$B'=\{V\subset X\times X\mid \text{on olemassa } U\in B\}$$ 
%$\U\subset\Pot (X\times X)$ karkein sellainen uniformiteetti, jolla $B\subset \U$.
%%jolla pätee seuraava ehto: Jos $V\in B$ ja $V\subset W\subset X\times X$ niin $ W\in\U_B$. 
%%Ehdosta seuraa $ B\subset\U_B$ ja että jokaiselle lähistölle $W\in\U$ löytyy 
%%kannan alkio $V\in B$, jolla pätee $V\subset W$. 
%%Näin ollen riittää tarkistaa, että $\U_B$ on uniformiteetti. 
Käydään läpi uniformiteetin määritelmän \ref{uniformi_maar} ehdot:
\begin{enumerate} 
\item[\ref{F_I}] Jos $V\in \U$ ja $V\subset W\subset X\times X$ niin $ W\in\U$,

Jos $V\in \U$ ja $V\subset W\subset X\times X$ niin $ W\in\U$,
\item[\ref{F_II}] Jokainen äärellinen leikkaus joukon $\U$ alkioista kuuluu joukkoon $\U$,
\item[\ref{U_I}] Vastaa ehtoa \ref{U'_I},
\item[\ref{U_II}] Jos $V\in B$, niin ehdon \ref{U'_II} nojalla on olemassa sellainen $V'\in B$, 
jolla $V'\subset V^{-1}$ ja näin ollen $V^{-1}\in\U_B$,
\item[\ref{U_III}] Vastaa ehtoa \ref{U'_III}.
\end{enumerate}
%TODO
\end{proof}
\end{lause}
\begin{lause}
%\emph{Uniformisen avaruuden topologia.}
Uniformisen avaruuden topologia.
Olkoon joukko $X$ varustettu uniformiteetilla $\U$.
Olkoon $x\in X$ 
alkio ja $V\in\U$ lähistö. 
%alkio, $V\in\U$ lähistö ja $B\subset\U$ uniformiteetin $\U$ kanta. 
Olkoon tällöin %$V(x)\subset X$,
\begin{equation*}V(x)=\{ y\in X\mid (x,y)\in V \}
\quad\text{ ja }\quad
B(x)=\{ V(x)\mid V\in\U \}
\end{equation*}
joukkoja.
Uniformiteetti $\U$ määrää topologian joukolle $X$ niin, 
että joukko $V(x)$ on (lähistön $V$ määräämä) ympäristö 
alkiolle $x$ ja joukko $B(x)$ on alkion $x$ 
ympäristökanta kyseisessä topologiassa.
\begin{proof}
Olkoon joukko $X$ varustettu uniformiteetilla $\U$, 
$x\in X$ alkio ja $V\in\U$ lähistö avaruudessa $X$. Lisäksi olkoot joukot $V(x)$ ja $B(x)$ kuten edellä. 
%
Joukko $V(x)$ on epätyhjä, sillä $x\in V(x)$.
%Alkio $x$ kuuluu joukkoon $V(x)$, joten joukko $V(x)$ on epätyhjä. 
%Jokaiselle lähistölle $V,W\in\U$ pätee 
%\begin{align*}
%V(x)\cup W(x) =&\{ y\in X\mid (x,y)\in V \text{ tai } (x,y)\in W \}\\
%=&\{ y\in X\mid (x,y)\in V\cup W \}
%\in B(x),
%\end{align*} 
%ja 
Olkoon $W\in\U$ lähistö, jolloin $W(x)$ on alkion $x$ ympäristö. 
Alkion $x$ ympäristöille $V(x)$ ja $W(x)$ pätee
\begin{align*}
V(x)\cap W(x) =&\{ y\in X\mid (x,y)\in V \text{ ja } (x,y)\in W \}\\
=&\{ y\in X\mid (x,y)\in V\cap W \}
\in B(x),
\end{align*} 
%sillä määritelmän \ref{uniformi_maar} ehtojen \ref{F_I} ja \ref{F_II} nojalla $ V\cup W \in B$ ja $ V\cap W \in B$. 
sillä määritelmän \ref{uniformi_maar} ehdon \ref{F_II} 
nojalla pätee $ V\cap W \in B$. 
%Olkoon lisäksi $U\subset X$ joukko, jolle pätee $V(x)\subset U$. 
%Nyt joukkoa $U$ vastaa jokin $U'\in X\times X$, 
%jolla pätee  $U'\in \U$.
Ehdosta \ref{F_I} seuraa, että 
joukko $B(x)$ on alkion $x$ kaikkien ympäristöjen joukko ja siten 
korollaarin \ref{kaikki_ystöt} nojalla ympäristökanta.
%Tällöin on olemassa yksikäsitteinen topologia, jossa $B(x)$ on alkion $x$ kaikkien ympäristöjen joukko.\cite{Eom1}
%yhdisteen kohdassa tarvitaan yleisempi tapaus. ehto V_4 puuttuu sivu19 Eom1
\end{proof}
\end{lause}
\begin{huom}
Uniformiteetin määräämää topologiaa sanotaan uniformiteetin indusoimaksi topologiaksi.
\end{huom}
\begin{maar}\label{tasaisesti_jatkuva}
%Uniformisesti jatkuvat kuvaukset. 
Olkoot $X$ ja $X'$ uniformeja avaruuksia 
ja $f\colon X\rightarrow X'$ kuvaus.
Kuvaus $f$ on \emph{tasaisesti jatkuva} (uniformly continuous), jos jokaiselle avaruuden $X'$ lähistölle $V'$ on olemassa avaruuden $X$ lähistö $V$ niin, että jokaiselle $(x,y)\in V$ pätee $(f(x),f(y))\in V'$. 
%Toisin sanoen jokaiselle avaruuden $X'$ lähistölle $V'$ on olemassa.
\end{maar}
\begin{kor}\label{uniformi alkukuva}
Olkoot $X$ ja $X'$ uniformeja avaruuksia 
ja $f\colon X\rightarrow X'$ tasaisesti jatkuva kuvaus. 
Olkoon $g\colon X\times X\rightarrow X'\times X'$ %sellainen 
kuvaus, jolla pätee $g(x,y)=(f(x),f(y))$ kaikilla $x,y\in X$. 
Tällöin jos $V'$ on avaruuden $X'$ lähistö, niin alkukuva $g^{\leftarrow}V'$ on avaruuden $X$ lähistö.
\begin{proof}
Korollaari on suora seuraus määritelmästä \ref{tasaisesti_jatkuva} ja uniformiteetin määritelmän \ref{uniformi_maar} ehdosta \ref{F_I}.
\end{proof}
\end{kor}
\begin{lause}
%Jokainen tasaisesti jatkuva kuvaus on jatkuva myös lähtö- ja maalijoukkojen uniformien määräämien topologioiden suhteen.
Olkoot $X$ ja $X'$ uniformeja avaruuksia 
ja $f\colon X\rightarrow X'$ tasaisesti jatkuva kuvaus. 
Tällöin kuvaus $f$ on jatkuva, %myös topologioiden kannalta, 
kun varustetaan joukot $X$ ja $Y$ uniformiteettiensa indusoimilla topologioilla.
\begin{proof}
Olkoot $X$ ja $X'$ uniformeja avaruuksia 
ja $f\colon X\rightarrow X'$ tasaisesti jatkuva kuvaus. 
Olkoon $g\colon X\times X\rightarrow X'\times X'$ %sellainen 
kuvaus, jolla pätee $g(x,y)=(f(x),f(y))$ kaikilla $x,y\in X$. 
Olkoon $V'$ avaruuden $X'$ lähistö ja $x'\in X'$ alkio. 
Avaruuden $X'$ uniformiteetin indusoimassa topologiassa kaava 
$$V'(x')=\{ y'\in X'\mid (x',y')\in V'\}$$
määrää alkion $x'$ ympäristön avaruudessa $X$.
Korollaarin \ref{uniformi alkukuva} mukaan lähistön $V'$ alkukuva $g^{\leftarrow}V'$ on avaruuden $X$ lähistö.
Olkoon nyt $x\in X$ sellainen alkio, jolla $f(x)=x'$.
Tällöin joukko $(g^{\leftarrow}V')(x)$ on alkion $x$ ympäristö. 
%Olkoon nyt $x\in g^\leftarrow (x')$ alkio, jolloin 
Erityisesti ympäristö $(g^{\leftarrow}V')(x)$ kuvautuu ympäristöön $V'(x')$, 
sillä ehdosta $(x,y)\in g^\leftarrow V'$ seuraa $(x',f(y))\in V' $ kaikilla $y\in X$.
%
%Tällöin olkoon $x\in X$ sellainen alkio, jolla $f(x)=x'$. 
%Nyt koska $g^{\leftarrow}V'$ on avaruuden $X$ lähistö, 
%niin $g^{\leftarrow}V'(x)$ on alkion $x'$ alkukuvaan $g^{\leftarrow}(x')$ kuuluvan alkion $x$ ympäristö. 
%
%Tällöin koska $g^{\leftarrow}V'$ on avaruuden $X$ lähistö, niin $g^{\leftarrow}V'(g^{\leftarrow}(x'))$ on alkion $x'$ alkukuvan $g^{\leftarrow}(x')\in X$ ympäristö. 
%erikseen 1-käs? eli alkukuva ympäristöstä lähistön sijaan?

Siis ympäristön alkukuva on ympäristö ja siten tasaisesti jatkuva kuvaus on jatkuva.
\end{proof}
\end{lause}
\begin{lause}
Olkoot $X$, $X'$ ja $X''$ uniformeja avaruuksia 
ja $f\colon X\rightarrow X'$ ja $g\colon X'\rightarrow X''$ 
tasaisesti jatkuvia kuvauksia. 
Tällöin yhdistetty kuvaus $g\circ f\colon X\rightarrow X''$ on tasaisesti jatkuva.
\begin{proof}
Olkoot $X$, $X'$ ja $X''$ uniformeja avaruuksia, 
$f\colon X\rightarrow X'$ ja $g\colon X'\rightarrow X''$ tasaisesti jatkuvia kuvauksia ja 
$V''$ avaruuden $X''$ lähistö. 
Tällöin tasaisesti jatkuvan kuvauksen määritelmän nojalla 
on olemassa avaruuden $X'$ lähistö $V'$, 
jolla jokaisella $(x',y')\in V'$ pätee $ (g(x'),g(y'))\in V''$.
Edelleen lähistölle $V'$ on olemassa avaruuden $X$ lähistö $V$, 
jolla jokaisella $(x,y)\in V$ pätee $ (f(x),f(y))\in V'$.
Näin ollen lähistölle $V''$ on olemassa lähistö $V$, 
jolla jokaisella $(x,y)\in V$ pätee $ (g(f(x)),g(f(y)))\in V''$, 
%ja koska yhdistetyllä kuvauksella pätee $g(f(x))=(g\circ f)(x)$, 
%niin jokaisella $(x,y)\in V$ pätee 
eli $ ((g\circ f)(x),(g\circ f)(y))\in V''$.
%Yhdistetyllä kuvauksella $g\circ f\colon X\rightarrow X''$ 
%pätee $(g\circ f)(x)$ 

Siis yhdistetty kuvaus $g\circ f\colon X\rightarrow X''$ on tasaisesti jatkuva.
\end{proof}
\end{lause}
\begin{maar}
Olkoot $X$ ja $X'$ uniformeja avaruuksia 
ja $f\colon X\rightarrow X'$ bijektiivinen kuvaus. 
Kuvaus $f$ on \emph{isomorfismi}, jos sekä kuvaus $f$ että sen 
käänteiskuvaus $f^{-1}$ ovat tasaisesti jatkuvia.
\end{maar}
\begin{maar}\label{uniformi_vertailu}
\emph{Uniformiteettien vertailu.} 
Olkoon $X$ joukko ja $\U_1$ ja $\U_2$ uniformiteetteja joukolle $X$. 
Uniformiteetti $\U_2$ on karkeampi kuin uniformiteetti $\U_1$, 
jos identtinen kuvaus $id\colon (X,\U_1)\rightarrow (X,\U_2)$ on tasaisesti jatkuva. Tällöin $\U_1$ on hienompi kuin $\U_2$. 

Jos lisäksi pätee $\U_1\neq\U_2$, niin $\U_1$ on aidosti hienompi kuin $\U_2$ ja vastaavasti $\U_2$ on aidosti karkeampi kuin $\U_1$. 
Sanotaan, että kahta uniformiteettia $\U_1$ ja $\U_2$ voidaan vertailla, 
jos $\U_1$ on hienompi tai karkeampi kuin $\U_2$. 
Uniformiteeteille $\U_1$ ja $\U_2$ pätee $\U_1=\U_2$, 
jos $\U_1$ on sekä hienompi että karkeampi kuin $\U_2$.
\end{maar}
\begin{kor}
Olkoon $X$ joukko ja $\U_1$ ja $\U_2$ uniformiteetteja joukolle $X$. 
Uniformiteetti $\U_1$ on hienompi kuin uniformiteetti $\U_2$ jos ja vain jos jokaisella lähistöllä $V\in\U_2$ pätee $V\in\U_1$.
\end{kor}
\begin{kor}
Olkoon $X$ joukko, $\U_1$ ja $\U_2$ uniformiteetteja joukolle $X$ ja 
$\U_1$ on hienompi kuin $\U_2$. 
Tällöin uniformiteetin $\U_1$ indusoima topologia on hienompi kuin 
uniformiteetin $\U_2$ indusoima topologia.
\end{kor}
\begin{maar}\label{kuvausperheen indusoima}
\emph{Kuvausperheen indusoima uniformiteetti} (initial uniformity).
Olkoon $X$ joukko ja $Y_i$ uniformiteetilla varustettuja joukkoja kaikilla $i\in I$, jollain indeksijoukolla $I$. 
%Olkoon $\U_i$ joukon $Y_i$ uniformiteetti kaikilla $i\in I$.
Olkoon $f_i\colon X\rightarrow Y_i$ kuvauksia kaikilla $i\in I$. 
Olkoon nyt $g_i=f_i\times f_i$ kuvaus joukolta $X\times X$ joukolle $Y_i\times Y_i$ 
%Olkoon $g_i=f_i\times f_i\colon X\times X\rightarrow Y_i\times Y_i$ %sellainen 
%kuvaus, jolla pätee $g(x,y)=(f_i(x),f_i(y))$ kaikilla $x,y\in X$. 
%kuvauksia 
kaikilla $i\in I$.
Olkoon 
$$B=\left\{\bigcap_{k=0}^{n}g^{\leftarrow}_{i_k}(V_{i_k})
\mid V_{i_k} \in\U_{i_k},i_k \in I,n\in\N \right\}$$
joukko missä $\U_{i_k}$ on avaruuden $Y_{i_k}$ uniformiteetti.
%Tällöin $B$ on kanta eräälle avaruuden $X$ uniformiteetille $\U$. 
Tällöin $B$ on kanta kuvausperheen $ (f_i)_{i\in I}$ 
avaruudelle $X$ indusoimalle uniformiteetille $\U$. 
Kyseinen uniformiteetti $\U$ on karkein niistä uniformiteeteista, joiden suhteen kaikki kuvaukset $f_i$ ovat tasaisesti jatkuvia.
\end{maar}
\begin{maar}
\emph{Uniformiteettien pienin yläraja.}
%Olkoon $X$ joukko%, $I$ indeksijoukko
% ja jokaisella $i\in I$ olkoon $\U_i$ uniformiteetti joukolle $X$. 
Olkoon $X$ joukko ja $I$ jokin indeksijoukko.
Olkoon $(\U_i)_{i\in I}$ perhe uniformiteetteja joukolle $X$.
Tällöin perheen $(\U_i)_{i\in I}$ \emph{pienin yläraja} on uniformiteetti $\U$, joka on kuvausten $id\colon X\rightarrow (X,\U_i)$ määritelmän \ref{kuvausperheen indusoima} mukaisesti indusoima.
\end{maar}
\chapter{Pseudometriikat}
Tässä kappaleessa tutustutaan pseudometriikoihin, jotka ovat tavanomaisten metriikoiden yleistys. Lisää aiheesta \cite{Eom2}.
%\\
\begin{maar}\label{pseudo_maar}
Olkoon $X$ joukko ja $f\colon X\times X\rightarrow [0,+\infty]$ 
% sellainen kuvaus, joka täyttää seuraavat ehdot:
kuvaus. Kuvaus $f$ on \emph{pseudometriikka}, jos seuraavat ehdot pätevät:
\begin{enumerate} [label=(P\arabic*),ref=(P\arabic*)]
\item\label{EC_I} $f(x,x)=0$ kaikilla $x\in X$,
\item\label{EC_II} $f(x,y)=f(y,x)$ kaikilla $x,y\in X$,
\item\label{EC_III} $f(x,y)\leq f(x,z)+f(z,y)$ kaikilla $x,y,z\in X$.
\end{enumerate}
\end{maar}
\begin{huom}
%Metriikka %$m\colon X\times X\rightarrow 
%on sellainen pseudometriikka, jolta lisäksi vaaditaan, että joukko $X$ on epätyhjä ja että 
Pseudometriikan määritelmästä saadaan metriikan määritelmä, jos rajoitutaan äärellisiin arvoihin ja vahvistetaan ehtoa \ref{EC_I} muotoon 
\begin{enumerate} %[label=(M\arabic*),ref=(M\arabic*)]
\item[(M1)]\label{M1} $f(x,y)=0\Leftrightarrow x=y$ kaikilla $x,y\in X.$
\end{enumerate}
Metriikan ehdot täyttävä kuvaus on pseudometriikka, 
joten %pseudometriikka on metriikan yleistys 
%ja siten 
jokainen metriikka on myös pseudometriikka.
\end{huom}
\begin{esim}%1)
Euklidinen etäisyys on pseudometriikka.
\end{esim}
\begin{esim}%2)
Olkoon $X$ epätyhjä joukko ja $f\colon X\times X\rightarrow [0,+\infty]$ sellainen kuvaus, jolla
\begin{equation*}
f(x,y) = \begin{cases} 0, & \mbox{jos } x=y\\
\infty, & \mbox{muulloin. } \end{cases}
\end{equation*}
Tällöin $f$ on pseudometriikka.
\end{esim}
\begin{esim}%3)
Olkoon $X$ epätyhjä joukko ja $g\colon X\rightarrow \R$ (äärellisarvoinen) kuvaus. Tällöin kuvaus $f\colon X\times X\rightarrow[0,+\infty]$ kaavalla $f(x,y)=|g(x)-g(y)|$ on pseudometriikka.
\end{esim}
\begin{esim}%4)
Olkoon $X$ kaikkien muotoa $g\colon [0,1]\rightarrow \R$ olevien jatkuvien kuvausten joukko. % väliltä $[0,1]$ reaaliluvuille. 
Tällöin kuvaus $f\colon X\times X\rightarrow [0,+\infty]$ kaavalla $f(x,y)=\int_0^1 |x(t)-y(t)|dt$ määrittelee pseudometriikan joukolle $X$.
\end{esim}
\begin{huom}
%Metriikoista tutut ominaisuudet, kuten
%Määritelmän \ref{pseudo_maar} ehdosta \ref{EC_III} seuraa, että jos $f(x,z)+f(z,y)<\infty$ niin $f(x,y)<\infty$. 
Määritelmän \ref{pseudo_maar} ehdon \ref{EC_III} nojalla epäyhtälöstä $f(x,z)+f(z,y)<\infty$ seuraa $f(x,y)<\infty$. 
%Tällöin koska kaavat $f(x,z)\leq f(x,y)+f(y,z)$ ja $f(y,z)\leq f(y,x)+f(x,z)$ pätevät, niin myös kaava $|f(x,z)-f(z,y)|\leq f(x,y)$ pätee.
Näin ollen epäyhtälöistä 
\begin{equation*}
f(x,z)\leq f(x,y)+f(y,z)\text{ ja }f(y,z)\leq f(y,x)+f(x,z)
\end{equation*}
%$f(x,z)\leq f(x,y)+f(y,z)$ ja $f(y,z)\leq f(y,x)+f(x,z)$ 
seuraa epäyhtälö $|f(x,z)-f(z,y)|\leq f(x,y)$.
\end{huom}
\begin{esim}%1
%Olkoon $f\colon X\times X\rightarrow$ pseudometriikka
Olkoon $f$ pseudometriikka joukolle $X$ ja 
$\lambda\in\R$ reaaliluku, jolla $\lambda >0$. 
Tällöin kuvaus $\lambda f$, jolla 
%$(\lambda f) (x,y)=\lambda (f (x,y))$ 
$(\lambda f) (x)=\lambda (f (x))$ 
%kaikilla $x,y\in X$ on pseudometriikka.
kaikilla $x\in X\times X$ on pseudometriikka.
%Tällöin myös $\lambda f$ on pseudometriikka, jos määritellään kuvaus $\lambda f$ pisteittäin $(\lambda f) (x)=\lambda (f (x))$.
\end{esim}
\begin{esim}%2
Olkoon $(f_i)_{i\in I}$ perhe joukon $X$ pseudometriikoita. 
Tällöin summakuvaus 
\begin{equation*}
f\colon X\times X\rightarrow [0,+\infty],\quad
%\text{ kaavalla } 
f(x,y)=\sum_{i\in I}f_i(x,y)%\text{ kaikilla }x,y\in X
\end{equation*} 
%$f\colon X\times X\rightarrow [0,+\infty]$ kaavalla $ f(x,y)=\sum_{i\in I}f_i(x,y)$ 
kaikilla $x,y\in X$ 
on pseudometriikka.
\end{esim}
\begin{esim}%3
Olkoon $(f_i)_{i\in I}$ perhe joukon $X$ pseudometriikoita. 
Tällöin kaikilla alkioilla $x,y\in X$ epäyhtälöstä 
$f_i(x,y)\leq f_i(x,z)+f_i(z,y)$ seuraa epäyhtälö 
$$\sup_{i\in I}f_i(x,y)\leq \sup_{i\in I}\left(f_i(x,z)+f_i(z,y)\right).$$
%\begin{align*}
%&f_i(x,y)\leq f_i(x,z)+f_i(z,y)\\
%\Rightarrow &\sup_{i\in I}f_i(x,y)\leq \sup_{i\in I}\left(f_i(x,z)+f_i(z,y)\right)
%\end{align*} 
Tällöin kuvaus 
$f\colon X\times X\rightarrow [0,+\infty]$
 missä  
%\quad\text{ kaavalla } 
\begin{equation*}
f(x,y)=\sup_{i\in I}f_i(x,y)
%$f(x,y)=\sup_{i\in I}f_i(x,y)$
\quad\text{kaikilla }x,y\in X 
%kaikilla $x,y\in X$
\end{equation*} 
on pseudometriikka. 
\end{esim}
\chapter{Pseudometriikan määrittelemä uniformiteetti}\label{luku_pseudo_uniformi}
%Oletamme koko luvun \ref{luku_pseudo_uniformi} ajan, että $\R_+=\{a\in\R\mid a>0\}$.
\begin{lause}
Olkoon $a\in\R_+$ reaaliluku ja $U_a\subset \R^n\times\R^n$ joukko kaavalla
$$U_a=\{ (x,y)\mid x,y\in\R^n,|x-y|\leq a\}.$$
Kokoelma $B=\{U_a\mid a\in\R_+\}$ muodostaa joukon $\R^n$ uniformiteetin kannan.
\begin{proof}
Lauseen \ref{uniformin kannan maar} ehdot pätevät:
\begin{enumerate} [label=(B\arabic*)]
\item %\ref{B_I} 
%Olkoon $a_1,a_2\in\R$ ja $a_3=\min(a_1,a_2)$ reaalilukuja. Tällöin $U_{a_3}\subset U_{a_1}\cap U_{a_2}$
Olkoon $a_1,a_2\in\R_+$ reaalilukuja, joilla $a_1\leq a_2$. Tällöin kaavasta $|x-y|\leq a_1$ seuraa $|x-y|\leq a_2$ kaikilla $x,y\in\R$. Näin ollen $\, U_{a_1}\subset U_{a_2}$ ja siis $U_{a_1}\subset U_{a_1}\cap U_{a_2}$,
%Jos $V_1,V_2\in B$ niin on olemassa sellainen $V_3\in B$, jolla $V_3\subset V_1\cap V_2$,
\item%\ref{U'_I} 
Olkoon $x\in\R^n$. Tällöin kaikilla $a\in\R_+$ pätee $|x-x|=0< a$, joten joukko $\{(x,x)\mid x\in X\}$ on jokaisen joukon $U_a\in B$ osajoukko,
%Joukko $\{(x,x)\mid x\in X\}$ on jokaisen joukon $V\in B$ osajoukko,
\item%\ref{U'_II} 
Olkoon $x',y'\in\R^n$ ja $a\in\R_+$. Tällöin jos $|x'-y'|\leq a$ niin myös $|y'-x'|\leq a$. Näin ollen 
%Olkoon $a\in\R_+$ ja $(x,y)\in U_a$. Tällöin koska $(x,y)\in U_a$ niin $|x'-y'|\leq a$ ja koska $U_a\subset \R^n\times\R^n$ niin $|y'-x'|\leq a$. Näin ollen 
\begin{align*}
U_a=&\{ (x,y)\mid x,y\in\R^n,|x-y|\leq a\}\\
=&\{ (y,x)\mid x,y\in\R^n,|x-y|\leq a\}\\
=&U_a^{-1}.
\end{align*}
Siis jokaiselle $U_a$ pätee $U_a\subset  U_a^{-1}. $
%Jos $V\in B$, niin on olemassa sellainen $V'\in B$, jolla $V'\subset V^{-1}%=\{(y,x)\mid (x,y)\in V\}$,
\item%\ref{U'_III} 
Olkoon $a\in\R_+$ reaaliluku ja $x,y,z\in\R^n$ sellaisia pisteitä, joilla $(x,z)\in U_a$ ja $(z,y)\in U_a$, eli $|x-z|\leq a$ ja $|z-y|\leq a$.
%Kolmioepäyhtälön $|x-y|\leq |x-z|+|z-y|$ nojalla . 
Tällöin kolmioepäyhtälön avulla saadaan 
$$|x-y|\stackrel{\triangle-\text{ey} }{\leq} |x-z|+|z-y|\leq a+a=2a,$$ 
eli $|x-y|\leq 2a$. Tästä seuraa, että $ U_a^2\subset U_{2a} $, joten jokaiselle reaaliluvun $b\in\R_+$ määräämälle joukolle $U_b$ löytyy joukko $U_{b/2}$, jolla $U_{b/2}^2\subset U_b$.
%Jos $V\in B$, niin on olemassa sellainen $W\in B$, jolla $ W^2\subset V$.%$ W\circ W\subset V$.
\end{enumerate}
Siis kokoelma $B=\{U_a\mid a\in\R_+\}$ muodostaa joukon $\R^n$ uniformiteetin kannan.
\end{proof}
\end{lause}
\noindent Edeltävää lausetta voidaan yleistää seuraavasti: 
\begin{lause}\label{pseudo_uniformi_maar}
Olkoon $X$ joukko ja $f$ pseudometriikka joukolle $X$. Tällöin pseudometriikka $f$ määrittelee sellaisen uniformiteetin joukolle $X$, jonka kannan muodostaa kokoelma $$B=\{ f^{\leftarrow}[0,a]\subset X\times X\mid a\in\R_+\}.$$
\begin{proof}
Olkoon $a\in\R_+$ reaaliluku ja merkitään 
joukkoa $f^{\leftarrow}[0,a]\subset X\times X$ kaavalla $f^{\leftarrow}[0,a]=U_a$. 
Lauseen \ref{uniformin kannan maar} ehdot pätevät:
\begin{enumerate} [label=(B\arabic*)]
\item %\ref{B_I} 
Olkoot $a_1,a_2\in\R_+$ reaalilukuja, joilla $a_1\leq a_2$. 
Tällöin pätee $$U_{a_1}=f^{\leftarrow}[0,a]\subset f^{\leftarrow}[0,b]=U_{a_2}$$ ja siis $U_{a_1}\subset U_{a_1}\cap U_{a_2}$,
%Jos $V_1,V_2\in B$ niin on olemassa sellainen $V_3\in B$, jolla $V_3\subset V_1\cap V_2$,
\item%\ref{U'_I} 
Kuvaus $f$ on pseudometriikka, joten pseudometriikan määritelmän \ref{pseudo_maar} ehdon \ref{EC_I} nojalla kaikilla $x\in X$ pätee $f(x,x)=0$. 
Toisin sanoen pistepareista $(x,x)$ muodostuvan joukon $\{(x,x)\mid x\in X\}$ jokainen alkio kuvautuu pseudometriikassa nollaksi. Näin ollen joukko $\{(x,x)\mid x\in X\}$ sisältyy jokaisen suljetun välin $[0,a]$ alkukuvaan $f^{\leftarrow}[0,a]$ kaikilla $a\in \R_+$.
%Joukko $\{(x,x)\mid x\in X\}$ on jokaisen joukon $V\in B$ osajoukko,
\item%\ref{U'_II} 
Pseudometriikan ehdosta \ref{EC_II} seuraa, että $f(x,y)=f(y,x)$ kaikilla $x,y\in X$. 
Tällöin $(x,y)\in U_a\Leftrightarrow (y,x)\in U_a$ ja siis $U_a^{-1}=U_a$, eli jokaiselle $U_a$ pätee $U_a\subset U_a^{-1}. $
%Jos $V\in B$, niin on olemassa sellainen $V'\in B$, jolla $V'\subset V^{-1}%=\{(y,x)\mid (x,y)\in V\}$,
\item%\ref{U'_III} 
Olkoon $a\in\R_+$ reaaliluku ja $x,y,z\in X$ sellaisia alkioita, joilla $(x,z)\in U_a$ ja $(z,y)\in U_a$, eli $f(x,z)\leq a$ ja $f(z,y)\leq a$.
%Kolmioepäyhtälön $|x-y|\leq |x-z|+|z-y|$ nojalla . 
Tällöin pseudometriikan ehdosta \ref{EC_III} seuraa 
$$f(x,y)\leq f(x,z)+f(z,y)\leq a+a=2a,$$ 
eli $f(x,y)\leq 2a$. Tästä seuraa, että $U_a^2\subset U_{2a} $, joten jokaiselle reaaliluvun $b\in\R_+$ määräämälle joukolle $U_b$ löytyy joukko $U_{b/2}$, jolla $U_{b/2}^2\subset U_b$.
%Jos $V\in B$, niin on olemassa sellainen $W\in B$, jolla $ W^2\subset V$.%$ W\circ W\subset V$.
\end{enumerate}
Siis kokoelma $B=\{U_a\mid a\in\R_+\}
=\{f^{\leftarrow}[0,a]\mid a\in\R_+\}
$ muodostaa joukon $\R^n$ uniformiteetin kannan ja voimme nyt muodostaa seuraavan määritelmän.
\end{proof}
\end{lause}
\begin{maar}\label{pseudo_equiv}
Olkoon $X$ joukko ja $f$ ja $g$ pseudometriikoita joukolle $X$. 
Tällöin lauseen \ref{pseudo_uniformi_maar} nojalla 
pseudometriikat $f$ ja $g$ määrittelevät jotkut uniformiteetit joukolle $X$.
%kokoelmat $B_f=\{ f^{\leftarrow}[0,a]\mid a\in\R_+\}$ ja $B_g=\{ g^{\leftarrow}[0,a]\mid a\in\R_+\}$ ovat uniformiteettien kantoja joukolle $X$. 
Pseudometriikat $f$ ja $g$ ovat \emph{ekvivalentteja}, jos ne määräävät saman uniformiteetin.
\end{maar}
\begin{kor}
Olkoon $X$ joukko ja $f$ ja $g$ pseudometriikoita joukolle $X$. 
Määritelmästä \ref{pseudo_equiv} seuraa, että pseudometriikan $f$ määräämä uniformiteetti 
$\U_f$ 
on karkeampi kuin pseudometriikan $g$ määräämä uniformiteetti $\U_g$, jos ja vain jos jokaiselle 
$a\in\R_+$ on olemassa sellainen $b\in\R_+$, jolla $g(x,y)\leq b \Rightarrow f(x,y)\leq a$ kaikilla $x,y\in X$.

Lisäksi, jos jokaiselle $a\in\R_+$ on olemassa $b\in\R_+$, jolla $f(x,y)\leq b \Rightarrow g(x,y)\leq a$ kaikilla $x,y\in X$, niin $f$ ja $g$ ovat ekvivalentteja pseudometriikoita.
\begin{proof}
Olkoon $X$ joukko ja $f$ ja $g$ pseudometriikoita joukolle $X$. 
Määritelmän \ref{pseudo_uniformi_maar} mukaan pseudometriikan $f$ määrittelemän uniformiteetin $\U_f$ kannan muodostaa kokoelma 
$$B_f=\{ f^{\leftarrow}[0,a]\subset X\times X\mid a\in\R_+\}.$$ 
Tällöin uniformiteetti $ \U_f$ on kokoelma 
$$\U_f=\{ V\subset X\times X\mid f^{\leftarrow}[0,a]\subset V, \text{ jollain } a\in\R_+\}.$$
Vastaavasti uniformiteetin $\U_g$ kannan muodostaa kokoelma $$B_g=\{ g^{\leftarrow}[0,a]\subset X\times X\mid a\in\R_+\}$$ ja uniformiteetti $ \U_g$ on kokoelma 
$$\U_g=\{ V\subset X\times X\mid g^{\leftarrow}[0,a]\subset V, \text{ jollain } a\in\R_+\}.$$ 
Osoitetaan väitteen implikaatio molempiin suuntiin.
\begin{enumerate}
\item[$\Rightarrow$] 
Olkoon uniformiteetti $\U_f$ karkeampi kuin $\U_g$ ja näin ollen määritelmän \ref{uniformi_vertailu} mukaan identtinen kuvaus $id\colon(X,\U_g)\rightarrow(X,\U_f)$ on tasaisesti jatkuva. 
Tällöin määritelmän \ref{tasaisesti_jatkuva} 
mukaan jokaiselle maalijoukon %avaruuden $(X,\U_f)$ 
lähistölle $V'\in\U_f $ on olemassa %avaruden $(X,\U_g)$ 
%sellainen lähistö $V\in\U_g $, 
lähtöjoukon lähistö $V\in\U_g $ niin, että 
%jolla seuraava ehto pätee: Jos $(x,y)\in V$, 
%jolla ehto $(x,y)\in V\Rightarrow (x,y)\in V'$ pätee kaikilla $x,y\in X$. 
jokaiselle $(x,y)\in V$ pätee $(x,y)\in V'$ kaikilla $x,y\in X$. 
%niin $(x,y)\in V'$ kaikilla $x,y\in X$. 
Tarkas\-tele\-mal\-la uniformiteettien kantojen jäseniä 
%saamme edeltävän ehdon seuraavaan muotoon: 
voidaan edeltävä ehto muotoilla seuraavasti: 
Jokaisella $a\in\R_+$ on olemassa $b\in\R_+$ niin, 
että kaikilla $x,y\in X$ pätee 
$(x,y)\in g^{\leftarrow}[0,b]\Rightarrow (x,y)\in f^{\leftarrow}[0,a]$, 
eli $g(x,y)\leq b\Rightarrow f(x,y) \leq a$.
\item[$\Leftarrow$] Oletetaan, että jokaiselle reaaliluvulle 
$a\in\R_+$ on olemassa sellainen $b\in\R_+$, jolla 
$g(x,y)\leq b \Rightarrow f(x,y)\leq a,
$ %\text{ 
eli %}
$(x,y)\in g^{\leftarrow}[0,b]\Rightarrow (x,y)\in f^{\leftarrow}[0,a],$ 
kaikilla $x,y\in X$. 
%Tällöin %kaikilla $x,y\in X$ pätee $(x,y)\in g^{\leftarrow}[0,b]\Rightarrow (x,y)\in f^{\leftarrow}[0,a]$, josta seuraa, että 
Siis jokaiselle uniformiteetin $\U_f$ kannan jäsenelle 
$f^{\leftarrow}[0,a']$ missä $a'\in\R_+ $ on olemassa uniformiteetin $\U_g$ 
kannan jäsen $g^{\leftarrow}[0,b']$ missä $b'\in\R_+$ niin, 
että kaikilla $x,y\in X$ pätee $(x,y)\in g^{\leftarrow}[0,b']\Rightarrow (x,y)\in f^{\leftarrow}[0,a']$. 
%että ehdosta $(x,y)\in g^{\leftarrow}[0,b']$ 
%%\Rightarrow 
%seuraa 
%$(x,y)\in f^{\leftarrow}[0,a']$, kaikilla $x,y\in X$. 
Näin ollen identtinen kuvaus $id\colon (X,\U_g)\rightarrow (X,\U_f)$ on 
tasaisesti jatkuva ja siten uniformiteetti $\U_g$ on uniformiteettia $\U_f$ hienompi. 
%Siis uniformiteetti $\U_f$ on karkeampi kuin uniformiteetti $\U_g$. 
Siis pseudometriikan $f$ määräämä uniformiteetti 
$\U_f$ 
on karkeampi kuin pseudometriikan $g$ määräämä uniformiteetti $\U_g$.
\end{enumerate}
Lisäksi 
määritelmistä \ref{uniformi_vertailu} ja \ref{pseudo_equiv} seuraa, että jos $\U_f$ on sekä hienompi että karkeampi kuin $\U_g$, niin pseudometriikat $f$ ja $g$ ovat ekvivalentteja.
\end{proof}
\end{kor}
\begin{maar}
Olkoon $X$ joukko ja $(f_i)_{i\in I} $ joukon $X$ pseudometriikkaperhe. 
Tällöin pseudometriikkojen $f_i$ määrittelemien uniformiteettien $\U_{f_i}$ pienintä ylärajaa sanotaan perheen $(f_i)_{i\in I}$ määrittelemäksi uniformiteetiksi. 
%uniformiteettien pienin yläraja
\end{maar}
\begin{maar}
Olkoon $X$ joukko ja olkoot $(f_i)_{i\in I} $ ja $(g_j)_{j\in J} $ joukon $X$ pseudometriikkaperheitä. 
Perheet $(f_i)$ ja $(g_j) $ ovat ekvivalentteja, jos niiden määrittelemät uniformiteetit ovat samoja.
\end{maar}
%\begin{esim}
%Olkoon $X$ joukko ja $(f_i)_{i\in I}$ mielivaltainen perhe (äärellisiä) reaaliarvoisia kuvauksia $f_i\colon X\rightarrow \R$ kaikilla $i\in I$.
%\end{esim}
\begin{maar}
Olkoon $X$ joukko ja $(f_i)_{i\in I} $ joukon $X$ pseudometriikkaperhe. 
Olkoon $H'\subset I$ äärellinen joukko ja $g_{H'}\colon X\times X\rightarrow [0,\infty]$ kuvaus, 
jolla $g_{H'}(x)=\sup_{i\in H'}f_i(x)$.
Nyt 
$$\{g_{H}^\leftarrow ([0,a])\subset X\times X\mid H\subset I \text{ äärellinen}, a\in\R_+ \}$$
on joukon $X$ erään uniformiteetin kanta. 
%Olkoon nyt $g_{H'}$ pseudometriikka jokaisella äärellisellä potenssijoukon osajoukolla $H'\subset\Pot(I)$. 
%Olkoon nyt $J=\{H\mid H\subset I\text{ äärellinen }\}$ 
Olkoon nyt $J=\{H\mid H\subset I, H\text{ äärellinen }\}$ 
potenssijoukon $\Pot(I)$ äärellinen osajoukko ja $g_{J}$ kuvaus, 
jolla $\sup_{H\in J}(g_H)\in (g_H)_{H\subset I} $. 

Nyt siis 
%$\sup_{H\in H'}(g_H)\in (g_H)_{H\subset I} $ 
$\sup_{H\in J}(g_H)\in (g_H)_{H\subset I} $ 
ja $H\subset I$ on äärellinen joukko.
Tällöin sanotaan, että perhe $(g_H)$ on \emph{saturoitu} (saturated), 
\emph{ekvivalentti} perheen $(f_i)_{i\in I}$ kanssa 
ja saatu saturoimalla perhe $(f_i)_{i\in I}$. 

Mikäli indeksijoukko $I$ on äärellinen, niin perheen $(f_i)_{i\in I}$ määräämä 
%uniformiteetti on sama kuin pseudometriikan $\sup_{i\in I} f_i$ määräämä.
uniformiteetti on sama kuin pseudometriikan $g=\sup_{i\in I} f_i$ määräämä.
\end{maar}
\begin{maar}\label{saturoitu}
Olkoon uniformiteetit $\U$ ja $\U'$ saturoitujen perheiden 
$(f_i)_{i\in I}$ ja $(g_j)_{j\in J}$ määräämiä. 
Uniformiteetti $\U$ on karkeampi kuin uniformiteetti $\U'$, 
jos jokaisella $i\in I$ ja $a\in\R_+$ löytyy $j\in J$ ja $b\in\R_+$, 
joilla ehdosta $g_j(x,y)\leq b$ seuraa $f_i(x,y)\leq a$. 
Vastaavasti tällöin uniformiteetti $\U'$ on hienompi kuin uniformiteetti $\U$.
\end{maar}
%Toisaalta on myös mahdollista määritellä i
\begin{lem}\label{pseudo_uniformista}
Olkoon $X$ joukko ja $\U$ uniformiteetti joukolle $X$. 
Tällöin on olemassa pseudometriikkaperhe $(f_i)_{i\in I}$, joka määrittelee uniformiteetin $\U$.
\begin{proof}
Jokaiselle lähistölle $U\in\U$ määritellään karteesisen tulon $X\times X$ osajoukoista muodostuva perhe $(V_n)$, 
jolla $V_1\subset U$ ja $V_{n+1}^2\subset V_n$ kaikilla $n\in\N, n\geq 1$. 
Nyt perhe $(V_n)$ on kanta eräälle joukon $X$ uniformiteetille $\U_V$, 
joka on karkeampi kuin $\U$. 
Erityisesti $\U$ on uniformiteettien 
$\U_V,V\in\U$ pienin yläraja. %uniformiteetti $\U$. 
Tällöin lemma on seuraus seuraavasta lauseesta, sillä 
$(U_n)$ on uniformiteetin $\U$ numeroituva kanta.
\end{proof}
\end{lem}
\begin{lause}
Olkoon $X$ joukko ja $\U$ uniformiteetti joukolle $X$. 
Jos uniformiteetilla $\U$ on numeroituva kanta, 
niin on olemassa pseudometriikka $f\colon X\times X\rightarrow \R_+$, 
jonka määräämä uniformiteetti on identtinen uniformiteetin $\U$ kanssa.
\begin{proof}
Olkoon $(V_n)$ numeroituva kanta uniformiteetille $\U$. 
Tällöin olkoon $(U_n)$ perhe symmetrisiä uniformiteetin $\U$ lähistöjä, 
joilla $U_1\subset V_1$ ja $U_{n+1}^3\subset U_n\cap V_n$, kun $n\geq 1$. 
Nyt $(U_n)$ on myös uniformiteetin $\U$ kanta ja erityisesti 
$U_{n+1}^3\subset U_n\cap V_n$, kun $n\geq 1$. \\
\\
Olkoon $g\colon X\times X\rightarrow \R_+$ kuvaus, jolla 
%$g(x,y)=0$, jos $(x,y)\in U_n$ kaikilla $n\geq 1$. 
%$g(x,y)=1$, jos $(x,y)\not\in U_1$. 
%$g(x,y)=0$, jos $(x,y)\in U_n$ kaikilla $n\geq 1$. 
\begin{equation*}
g(x,y)=
\begin{cases}
0&\text{, jos }(x,y)\in U_n\text{ kaikilla }n\geq 1, \\
1&\text{, jos }(x,y)\not\in U_1, \\
2^{-k}&\text{, jos }(x,y)\in U_n\text{ kaikilla }n\leq k 
\text{ ja } (x,y)\not\in U_{k+1}.
\end{cases}
\end{equation*}
Nyt $g$ on symmetrinen, positiivinen ja $g(x,x)=0$ kaikilla $x\in X$. 

Olkoon nyt $f\colon X\times X\rightarrow \R_+$ kuvaus, 
jolla 
$$f(x,y)=\inf\sum_{i=0}^{p-1}g(z_i,z_{i+1}),$$
missä $p\in\N,p\geq 1$ ja $(z_i)_{0\leq i\leq p}$ 
joukon $X$ alkioista muodostuva jono, jossa $z_0=x$ ja $z_p=y$. 
Kuvauksen $f$ määrittelystä seuraa, 
että $f$ on symmetrinen, kolmioepäyhtälö pätee ja kaavat 
$f(x,y)\geq 0$ ja $f(x,x)=0$ pätevät kaikilla $x,y\in X$.
Siis $f$ on pseudometriikka. 
Näytetään seuraavaksi, että epäyhtälöt 
\begin{equation}\label{numeroituvakantapm}
\frac{1}{2}\,g(x,y)\leq f(x,y)\leq g(x,y)
\end{equation}
pätevät. 
Kaavan oikea puoli, eli $f(x,y)\leq g(x,y)$ seuraa siitä, 
että 
$$f(x,y)=\inf\sum_{i=0}^{p-1}g(z_i,z_{i+1})\leq\sum_{i=0}^1g(z_i,z_i+1)=g(z_0,z_1)=g(x,y).$$
Kaavan \ref{numeroituvakantapm} vasen puoli, 
eli $\frac{1}{2}\,g(x,y)\leq f(x,y)$ 
osoitetaan induktion avulla. 
Olkoon $p\in\N$ luonnollinen luku. 
Nyt jokaisella $p+1$ alkion jonolla joukon $X$ 
alkioita $ (z_i)_{0\leq i\leq p}$, 
jolla $z_0=x$ ja $z_p=y$, saadaan induktio-oletukseksi 
\begin{equation}\label{numeroituvakantapm2}
\sum_{i=0}^{p-1}g(z_i,z_{i+1})\geq \frac{1}{2}\,g(x,y).
\end{equation}
%Tämä kaava . 
Jos $p=1$, niin summassa on vain yksi termi
$$g(z_0,z_{1})=g(x,y)\geq \frac{1}{2}\,g(x,y).$$
Merkitään 
\begin{equation}\label{numeroituvakantapm3}
a=\sum_{i=0}^{p-1}g(z_i,z_{i+1}),
\end{equation}
jolloin induktio-oletus voidaan kirjoittaa 
muodossa $a\geq \frac{1}{2}g(x,y) $. 
%$\frac{1}{2}g(x,y)\leq a $.
Määrittelyn nojalla $g(x,y)\leq 1$, joten jos $a\geq 1/2$, 
niin yhtälö \ref{numeroituvakantapm2} pätee muodossa 
$a\geq \frac{1}{2}\geq \frac{1}{2}g(x,y)$. 
Oletetaan, että $a<1/2$ ja että $h$ on suurin niistä indekseistä $q$, 
joilla $\sum_{i<q}g(z_i,z_{i+1})\leq a/2$. 
Tällöin 
\begin{equation*}\label{numeroituvakantapm4}
\sum_{i<h}g(z_i,z_{i+1})\leq a/2\quad\text{ ja lisäksi }\quad
\sum_{i>h}g(z_i,z_{i+1})\leq a/2,
\end{equation*}
sillä $\sum_{i<h+1}g(z_i,z_{i+1})> a/2$. 
%Nyt siis $\frac{1}{2}g(x,y)\leq a$, 
Edeltävien kaavojen ja induktio-oletuksen nojalla 
(sijoituksella $p=h$) pätee
\begin{equation*}
\frac{1}{2}a\geq \sum_{i<h}g(z_i,z_{i+1})\geq \frac{1}{2}\,g(x,z_h),
\end{equation*}
eli $g(x,z_h)\leq a$. 
Toisaalta yllä olevien kaavojen nojalla pätee myös 
%sijoituksella "$0=h+1$"
\begin{equation*}
\frac{1}{2}a\geq \sum_{i>h}g(z_i,z_{i+1})\geq \frac{1}{2}\,g(z_{h+1},y),
\end{equation*}
eli $g(z_{h+1},y)\leq a$. 
%Tällöin kaavojen %\ref{numeroituvakantapm2}, 
%\ref{numeroituvakantapm3} ja \ref{numeroituvakantapm4} nojalla 
%Näin ollen $g(x,z_h)\leq a$, $g(z_{h+1},y)\leq a$ ja 
Toisaalta luvun $a$ 
%on summa positiivisista luvuista, joten 
määrittelyn nojalla 
$g(z_h,z_{h+1})\leq a$. 

Olkoon $k\in \N$ pienin luku, jolla $2^{-k}\leq a$. 
Tällöin oletuksesta $a<1/2$ seuraa $k\geq 2$ ja kuvauksen $g$ määrittelystä seuraa, että 
$(x,z_h)\in U_k,(z_h,z_{h+1})\in U_k$ ja $(z_{h+1},y)\in U_k$. 
%TODO tästä eteenpäin vaatii viimeistelyä
%Näin ollen 
Nyt 
%eli 
$(x,y)\in U_k^3$ ja 
lähistöperheen $(U_n)$ määrittelyn nojalla $ (x,y)\in U_{k-1}$. 
%$ U_k^3\subset U_{k-1}$ 
Kuvauksen $g$ määrittelyn nojalla 
$g(x,y)\leq 2^{-(k-1)}=2^{1-k}\leq 2a$, eli $\frac{1}{2}g(x,y)\leq a.$ 

Näin ollen kaavan \ref{numeroituvakantapm} epäyhtälöt pätevät ja 
niistä seuraa, että jokaisella $a>0$ pätee 
$U_k\subset f^\leftarrow([0,a])$, kun $2^{-k}<a$. 
Toisaalta myös $f^\leftarrow([0,a])\subset U_k$, 
joten joukot $f^\leftarrow([0,a])$ muodostavat kannan uniformiteetille $\U$. 
Siis löydettiin pseudometriikka $f$, joka määrittelee uniformiteetin $\U$.
\end{proof}
\end{lause}
%\begin{kor}
%\end{kor}
%
%
\chapter{Uniformisoituvat avaruudet}
Tässä luvussa perehdytään uniformisoituviin (uniformizable) avaruuksiin. 
\begin{maar}
Olkoon $(X,\T)$ topologinen avaruus ja 
$\U$ uniformiteetti joukolle $X$. 
%Merkitään uniformiteetin uniformiteetin $\U$ indusoimaa topologiaa $\T_{\U}$. 
Uniformiteetti $\U$ on \emph{yhteensopiva} topologian $\T$ kanssa, 
jos uniformiteetin $\U$ indusoima topologia 
$\T_{\U}$ on sama kuin topologia $\T$.
\end{maar}
%\begin{kor}
%
%\end{kor}
\begin{maar}
Topologinen avaruus $(X,\T)$ on \emph{uniformisoituva}, jos joukolle $X$ voidaan muodostaa topologian $\T$ kanssa yhteensopiva uniformiteetti.
\end{maar}
\begin{lause}
Olkoon $(X,\T)$ topologinen avaruus.
%Topologinen avaruus $(X,\T)$ on uniformisoituva, jos seuraava ominaisuus pätee:
Avaruuden $(X,\T)$ uniformisoituvuus on yhtäpitävää seuraavan ehdon kanssa:
%\begin{itemize}
\begin{enumerate} [label=(Z),ref=(Z)]
\item\label{O_IV} Kaikkien alkioiden $x\in X$ kaikilla 
ympäristöillä $V\subset X$ on olemassa jatkuva 
reaaliarvoinen kuvaus $f\colon X\rightarrow [0,1]$, 
jolla $f(x)=0$ ja $f(y)=1$ kaikilla $y\in X\setminus V$.
%\end{itemize}
\end{enumerate}
\begin{proof}
Näytetään ensin, että ehto on välttämätön. 
Lauseen \ref{pseudo_uniformista} nojalla uniformiteetille voidaan 
määritellä pseudometriikkaperhe, joka indusoi kyseisen uniformiteetin. 
Lisäksi uniformiteetin indusoiman pseudometriikkaperheen voidaan olettaa määritelmän \ref{saturoitu} 
nojalla olevan saturoitu. 
Tällöin määritelmän \ref{pseudo_uniformi_maar} nojalla on olemassa $ a\in\R_+$, $\alpha\in I$ 
ja pseudometriikka $f_\alpha\in (f_i)_{i\in I}$, 
jolla $f_\alpha (x_0,x)\geq a$ kaikilla $x\in X\setminus V_0$ ja $f_\alpha(x_0,x)=0$. 
Tämän seurauksena kuvaus $g\colon X\to [0,1]$ kaavalla $g(x)=\inf \left( 1,\frac{1}{a}f_\alpha(x_0,x) \right)$ 
pisteelle $x_0$ ja ympäristölle $V_0$ toteuttaa ehdon \ref{O_IV}.

Toisaalta ehto on myös riittävä:
Olkoon $(X,\T)$ topologinen avaruus, jolla ehto \ref{O_IV} pätee. 
Olkoon $x_0\in X$ alkio, $V_0\subset X$ alkion $x_0$ ympäristö 
ja $f\colon X\rightarrow [0,1]$ ehdon \ref{O_IV} antama kuvaus alkiolle $x_0$ ja sen ympäristölle $V_0$. 
Tällöin kuvaus $g\colon X\times X\to [0,\infty]$ kaavalla $g(x_0,x)=f(x)$ on pseudometriikka. 
%Olkoon $(g_i)_{i\in I}$ näin muodostettujen pseudometriikkojen perhe ja olkoon $\U$ kyseisen pseudometriikkaperheen määräämä uniformiteetti.
%, jonka määrää pseudometriikkaperhe $(g_i)_{i\in I}$ joukon $X$.
Olkoon $\U$ pseudometriikan $g$ määräämä uniformiteetti ja 
$ a\in R_+$ reaaliluku. 
Tällöin $g^\leftarrow[0,a]=\{(x,y)\mid g(x,y)\leq a\}$ on 
lähistö pseudometriikan $g$ määräämässä uniformiteetissa $\U$. 
%Merkitään $U_a=g^\leftarrow[0,a]$ joukko.
%Merkitään joukkoa $g^\leftarrow[0,a]$ kaavalla $U_a$, jolloin  
Merkitään $g^\leftarrow[0,a]=U_a$, jolloin  
%Nyt joukko $$U_a(x)=\{y\mid (x,y)\in U_a\}=\{y\mid g(x,y)\leq a\}$$ on alkion $x$ ympäristö uniformiteetin $\U$ 
joukko $U_a(x_0)=\{y\mid g(x_0,y)\leq a\}$ on alkion $x_0$ ympäristö uniformiteetin $\U$ 
indusoimassa topologiassa $\T'$. 
Joukkojen määrittelyistä seuraa, että jos $a<1$, niin $U_a(x)\subset V_0$ ja tällöin kokoelma $B=\{U_a(x_0)\mid g(x_0,y)\leq a\}$ 
on alkion $x_0$ ympäristökanta topologiassa $\T$.
%Toisaalta kokoelma $B=\{U_a\mid a\in\R_+ \}$ muodostaa kannan uniformiteetille $\U$ 
%ja tällöin kokoelma $B(x)=\{U_a(x)\mid a\in\R_+ \}$ on alkion $x$ ympäristökanta topologiassa $\T'$. 
%
%Toisaalta topologian $\T$ ympäristökanta on myös 

Siis avaruus $(X,\T)$, joka toteuttaa ehdon \ref{O_IV} on uniformisoituva.
\end{proof}
\end{lause}
%
%
\chapter{Täydellinen uniforminen avaruus}
Tässä luvussa määritellään täydellinen uniforminen avaruus minimaalisten Cauchy filtterien avulla.

Sellaiselle joukolle, jolle on määritelty uniformiteetti voidaan määritellä myös 
\emph{mielivaltaisen pieni} osa\-jouk\-ko kyseisen uniformiteetin suhteen. 
Osajoukko on mielivaltaisen pieni silloin, kun kaikki osajoukon alkiot ovat 
mielivaltaisen lähellä toisiaan. 
Näin saadaan seuraava määritelmä.
\begin{maar}
Olkoon $X$ joukko, $\U$ uniformiteetti joukolle $X$ ja $V\in\U$ lähistö. 
Osajoukko $A\subset X$ on \emph{$V-$pieni}, jos $(x,y)\in V$ jokaisella $x,y\in X$, eli jos $A\times A\subset V$.
\end{maar}
\begin{lause}
Olkoon $X$ joukko, $\U$ uniformiteetti joukolle $X$ ja $V\in\U$ lähistö. 
Olkoon lisäksi $A,B\subset X$ $V-$pieniä osajoukkoja, 
joiden leikkaus on epätyhjä. 
Tällöin yhdiste $A\cup B$ on $V^2-$pieni.
\begin{proof}
Olkoot $x\in A$, $y\in B$ ja $z\in A\cap B$ alkioita. 
Tällöin alkioparit $(x,z)$ ja $(z,y)$ kuuluvat lähistöön $V$ 
ja näin ollen alkiopari $(x,y)$ kuuluu joukkoon $V^2$.
\end{proof}
\end{lause}
\begin{maar}
\emph{Filtteri.} Olkoon $X$ joukko ja $\F\subset \Pot(X)$ potenssijoukon osajoukko. 
Joukko $F$ on \emph{filtteri} joukolle $X$, jos sille pätee 
uniformiteetin ehtojen \ref{F_I} ja \ref{F_II} lisäksi 
seuraava ehto:
\begin{enumerate}
\item[(F)] Tyhjä joukko ei kuulu joukkoon $\F$.
\end{enumerate} 
%Tällöin sanotaan, että
\end{maar}
\begin{maar}
Olkoon $X$ uniforminen avaruus. Joukko $B\subset\Pot(X)$ on erään filtterin $\F$ kanta, jos seuraavat ehdot pätevät:
\begin{enumerate} 
\item Jos $A_1,A_2\in B$, niin on olemassa sellainen $A_3\in B$, 
jolla $A_3\subset A_1\cap A_2$.
\item joukko $B$ on epätyhjä eikä tyhjä joukko ole joukon $B$ alkio.
\end{enumerate}
Tällöin kanta $B$ virittää filtterin $\F$. 
%Lisäksi filttereiden $\F_1$ ja $\F_2$ kannat $B_1$ ja $B_2$ ovat 
%\emph{ekvivalentteja}, jos $\F_1=\F_2$.
Lisäksi kaksi kantaa ovat 
\emph{ekvivalentteja}, jos ne virittävät saman filtterin.
\end{maar}
\begin{lause}
Joukon $X$ filtteri $\F$ kannalla $B$ on hienompi kuin 
filtteri $\F'$ kannalla $B'$ jos ja vain jos jokaiselle $M\in B$ löytyy 
sellainen $M'\in B'$, jolla $M'\subset M$.
\end{lause}
\begin{maar}\label{Cauchy_filtteri}
Olkoon $X$ joukko, $\U$ uniformiteetti joukolle $X$ ja $\F$ filtteri joukolle $X$. 
Filtteriä $\F$ sanotaan \emph{Cauchy} filtteriksi, 
jos jokaiselle lähistölle $V\in\U$ löytyy osajoukko $A\subset X$, 
joka on $V-$pieni ja kuuluu filtteriin $\F$. %$A\in F$. 
\\
\\
Cauchy filtterit ovat siis mielivaltaisen pieniä joukkoja. 
\end{maar}
\begin{maar}
\emph{Filttereiden vertailu.}
Olkoon $X$ joukko ja $\F_1$ ja $\F_2$ filttereitä joukolle $X$. 
Filtteri $\F_2$ on karkeampi kuin filtteri $\F_1$, jos $\F_2\subset \F_1$. 
Tällöin $\F_1$ on hienompi kuin $\F_2$. 

Jos lisäksi pätee $\F_1\neq\F_2$, niin $\F_1$ on aidosti hienompi kuin $\F_2$ ja vastaavasti $\F_2$ on aidosti karkeampi kuin $\F_1$. 
Sanotaan, että kahta filtteriä $\U_1$ ja $\U_2$ voidaan vertailla, 
jos $\F_1$ on hienompi tai karkeampi kuin $\F_2$. 
Filttereille $\F_1$ ja $\F_2$ pätee $\F_1=\F_2$, 
jos $\F_1$ on sekä hienompi että karkeampi kuin $\F_2$.
\end{maar}
\begin{lause}
Topologisessa avaruudessa osajoukon (vastaavasti alkion) kaikkien ympäristöjen joukko on filtteri, jota sanotaan ympäristöfiltteriksi.
\end{lause}
\begin{maar}
\emph{Filtterin raja-arvo.} 
Olkoon $(X,\T)$ topologinen avaruus ja $\F$ filtteri joukolle $X$. 
Alkio $x\in X$ on filtterin $\F$ \emph{raja-arvo}, 
jos filtteri $\F$ on hienompi kuin 
alkion $x$ ympäristöfiltteri. % $B(x)$. 
Näin ollen filtteri $\F$ \emph{suppenee} (converge) kohti alkiota $x$. 

Lisäksi, jos $B$ on filtterin $\F$ kanta 
ja alkio $x$ on filtterin $\F$ raja-arvo, 
niin alkio $x$ on myös kannan $B$ raja-arvo ja 
kanta $B$ suppenee kohti alkiota $x$.
\end{maar}
\begin{lause}
Filtterikanta $B$ suppenee kohti alkiota $x\in X$, jos ja vain jos 
jokainen alkion $x$ ympäristö sisältää filtterikannan jäsenen.
\begin{proof}
%TODO
\end{proof}
\end{lause}
\begin{lause}
Filtteri $\F$ suppenee kohti alkiota $x\in X$, 
jos ja vain jos jokainen 
filtteriä $\F$ hienompi ultrafiltteri suppenee kohti alkiota $x$.
\begin{proof}
%TODO
\end{proof}
\end{lause}
\begin{lause}
Olkoon $X$ uniforminen avaruus. Tällöin jokainen suppeneva filtteri on Cauchy filtteri.
\begin{proof}
Olkoon $X$ joukko, $\F$ suppeneva filtteri joukolle $X$ ja $x\in X$ 
sellainen alkio, jota kohti $\F$ suppenee. 
Tällöin alkion $x$ ympäristöfiltteri $B(x)$ sisältyy filtteriin $\F$. 
Näin ollen jokaisella lähistöllä $V'\in\U$ pätee $V'\in\F$. 

Olkoon nyt $V\in\U$ lähistö ja $x\in X$ alkio. Ehdon \ref{Uaehto} nojalla on olemassa 
sellainen lähistö $W\in\U$, jolla $W^{-1}\circ W\subset V$. 
Huomataan, että joukko 
$$W(x)\times W(x)=\{ (y_1,y_2)\mid (x,y_1)\in W,(x,y_2)\in W\} $$
sisältyy joukkoon 
$$W^{-1}\circ W=\{ (y_1,y_2)\mid \text{on olemassa }x',\text{ jolla }(y_1,x')\in W^{-1},(x',y_2)\in W\}.$$ 
Näin ollen lähistölle $V$ löydettiin osajoukko $W(x)\in X$, 
jolla $W(x)\times W(x)\subset W^{-1}\circ W\subset V$ ja $W(x)\in\F$. 
Määritelmän \ref{Cauchy_filtteri} nojalla $\F$ on Cauchy filtteri.
\end{proof}
\end{lause}
\begin{lause}
\emph{Minimaalinen Cauchy filtteri.} 
Olkoon $(X,\U)$ uniforminen avaruus. 
Jokaiselle joukon $X$ Cauchy filtterille on olemassa yksikäsitteinen 
minimaalinen Cauchy filtteri $\F_0$, joka on karkeampi kuin $\F$, 
eli $\F_0\subset \F$. 
Jos $B$ on kanta filtterille $\F$ ja $G$ on uniformiteetin $\U$ symmetristen lähistöjen kanta, 
niin joukot $V(M)=\{ y\mid (x,y)\in V,x\in M \}$ missä $M\in B$ ja $V\in G$ muodostavat 
filtterin $\F$ minimaalisen Cauchy filtterin kannan $B_0$. 
%niin joukko $\{V(M)\mid M\in B, V\in G \}$ muodostaa 
%filtterin $\F$ minimaalisen Cauchy filtterin kannan $B_0$. 
%niin joukko $\{y\mid (x,y)\in V,x\in M, M\in B, V\in G \}$ muodostaa 
%filtterin $\F$ minimaalisen Cauchy filtterin kannan $B_0$. 
%$V(M)=\{ y\mid (x,y)\in V,x\in M \}$
\begin{proof}
%Lauseessa määritelty $B_0$ on kanta, sillä kannan ehdot pätevät: 
Tarkistetaan kannan ehdot joukolle $B_0$:
\begin{enumerate}
\item Olkoot $V_1(M_1),V_2(M_2)\in B_0$ joukkoja. 
Nyt $M_1,M_2\in B$, joten on olemassa joukko $M_3\in B$, 
jolla pätee $M_3\subset M_1\cap M_2$. 
Toisaalta $V_1,V_2\in \U$, joten on olemassa joukko $V_3\in\U$, 
jolla pätee $V_3\subset V_1\cap V_2$. 
Nyt joukko $ V_3(M_3)=\{ y\mid (x,y)\in V_3, x\in M_3 \}$ on joukon 
$(V_1\cap V_2)(M_1\cap M_2)= V_1(M_1)\cap V_2(M_2)$ osajoukko, 
eli $ V_3(M_3)\subset V_1(M_1)\cap V_2(M_2)$.
\item Olkoon $M\in B$ joukko ja $V\in \U$ lähistö. 
Tällöin joukko $M$ on epätyhjä, joten myös joukko $V(M)$ on epätyhjä. 
Näin ollen tyhjä joukko ei ole joukon $B_0$ alkio.
\end{enumerate}
Kannan $B_0$ virittämä filtteri $\F_0$ on karkeampi kuin kannan $B$ virittämä $\F$, 
sillä jokaisella $M\in B$ ja jokaisella $V\in \U$ pätee $M\subset V(M)$. 
%TODO 1-käs
\end{proof}
\end{lause}
\begin{maar}

\end{maar}
%
%
\chapter{Hausdorff uniforminen avaruus}
Tässä luvussa määritellään Hausdorff uniformiset avaruudet ja 
esitetään niiden ominaisuuksia, 
oleellisimpana täydelliseen Hausdorff uniformiseen avaruuteen laajentaminen.
\begin{maar}
Olkoon $X$ joukko ja $\U$ uniformiteetti joukolle $X$. 
Uniformiteetti $\U$ on \emph{Hausdorff}, jos kaikille $x,y\in X,x\neq y$ on 
olemassa pseudometriikka $f_i\in(f_i)_{i\in I}$, jolla $f_i(x,y)\neq 0$. 
Erityisesti, jos uniformiteetti $\U$ on Hausdorff 
ja yhden pseudometriikan $f$ määrittelemä, 
niin $f(x,y)\neq 0$ kaikilla $x,y\in X$.

Uniformiteetti $\U$ ei ole Hausdorff, jos on olemassa sellaiset alkiot $x,y\in X$, joilla $x\neq y$ ja $f_i(x,y)=0$ kaikilla $i\in I$.
\end{maar}
\begin{lem}
Olkoon $X$ uniforminen avaruus, $A\subset X$ epätyhjä osajoukko 
ja $f$ pseudometriikka joukolle $X$. 
Tällöin pseudometriikan rajoittuma 
$f|_A\colon A\times A\rightarrow [0,\infty]$ kaavalla $ f|_A(x)=f(x)$ 
kaikilla $x\in A\times A$ 
on pseudometriikka joukolle $A$. 
\end{lem}
\begin{lem}
Olkoon $X$ uniforminen avaruus, $A\subset X$ epätyhjä osajoukko 
ja $(f_i)_{i\in I}$ pseudometriikkaperhe, 
joka määrää joukon $X$ uniformiteetin. 
Tällöin joukon $X$ uniformiteetti määrää joukolle $A$ saman uniformiteetin 
kuin pseudometriikkaperhe $(f_i|_A)_{i\in I}$.
\end{lem}
\begin{maar}\label{complete}
Olkoon $X$ uniforminen avaruus. 
Tällöin on olemassa täydellinen (com\-plete) Hausdorff 
uniforminen avaruus $\hat X$ ja tasaisesti jatkuva 
kuvaus $i\colon X\rightarrow\hat X$, jolle pätee seuraava ominaisuus:
\begin{enumerate} [label=(P),ref=(P)]
%\begin{enumerate}
\item %[(P)] 
\label{ominaisuusP}
Olkoon $Y$ täydellinen Hausdorff uniforminen avaruus 
ja $f\colon X\rightarrow Y$ tasaisesti jatkuva kuvaus. 
Tällöin on olemassa yksikäsitteinen tasaisesti 
jatkuva $g\colon \hat X\rightarrow Y$, 
jolla pätee $f=g\circ i$
\end{enumerate}
%\end{maar}
%\begin{maar}
Olkoon lisäksi $X_1$ täydellinen Hausdorff 
uniforminen avaruus ja $i_1\colon X\rightarrow X_1$
tasaisesti jatkuva kuvaus, jolla on ominaisuus \ref{ominaisuusP}. 
Tällöin on olemassa yksikäsitteinen isomorfismi $\varphi\colon\hat X\rightarrow X_1$, jolla pätee $i_1=\varphi\circ i$.
\end{maar}
\begin{kor}
Olkoon $X$ on Hausdorff uniforminen avaruus ja $\hat X$ määritelmän \ref{complete} mukainen täydellinen Hausdorf uniforminen avaruus. 
Niin sanottu kanoninen kuvaus $i\colon X\rightarrow\hat X$ määrää 
isomorfismin $X\rightarrow X'$, jossa $X'\subset \hat X$ on tiheä joukossa $\hat X$.
\end{kor}


%\begin{huom}
%\end{huom}
%\begin{lause}
%\end{lause}
%\begin{maar}
%\end{maar}
%\begin{kor}
%\end{kor}
%\begin{esim}
%\end{esim}
%\begin{lem}
%\end{lem}
\begin{thebibliography}{9}

\bibitem{Eom1}
Nicolas Bourbaki: General Topology Part 1, 1.\ painos, Hermann, 1966.

\bibitem{Eom2}
Nicolas Bourbaki: General Topology Part 2, 1.\ painos, Hermann, 1966.

%\bibitem{Bor}
%Karol Borsuk: Theory of retracts, %n.\ painos, 
%Państwowe Wydawn. Naukowe, 1967.
%
%\bibitem{Topo1}
%Jussi Väisälä: Topologia I, 4.\ korjattu painos, Limes ry, 2007.
%
\bibitem{Topo2}
Jussi Väisälä: Topologia II, 2.\ korjattu painos, Limes ry, 2005.

%\bibitem{Hei}
%Juha Heinonen: Geometric embeddings of metric spaces, luentomoniste, Jyväskylän yliopisto, 2003
%Sheldon (not really) Ross: A First Course in Probability, 5th edition, Prentice-Hall, 1998.

%\bibitem{Tuo}
%Pekka (not really) Tuominen: Todennäköisyyslaskenta I, 5.\ painos, Limes ry, 2000.

\end{thebibliography}

\end{document}
