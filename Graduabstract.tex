%%
% HELSINGIN YLIOPISTON  opinnäytteisiin liitettävän tiivistelmäsivun
% LaTeX2e-kielinen  kaavakemäärittely.
%
% Laatinut:  Greger Linden
%
% Modif. :   Juha Korpi   (tiivistelmäteksti parbox-määrittelyn avulla)
%
% Modif. :   Tiina Niklander  (muunnos SLaTeX -> LaTeX)
%
% Modif  :   Tapio Lehtonen (CR-luokitukset alareunaan)
%                           (englanninkieliset kentännimet 18041994)
%                           (Viimeinenkin 7-bittisyys pois 09061994)
% Hieman muokannut Martti Nikunen

\documentclass[a4paper]{article}
\usepackage[utf8]{inputenc}
\usepackage[T1]{fontenc}
\usepackage[finnish]{babel}
\renewcommand{\baselinestretch}{1.2}
\def\division{\char'057}
\setlength{\unitlength}{.95pt}
\setlength{\textwidth}{21cm}
\setlength{\textheight}{29.7cm}
\setlength{\topskip}{0cm}
\setlength{\footskip}{0cm}
\addtolength{\oddsidemargin}{-2.0\oddsidemargin}
\addtolength{\topmargin}{-3.5cm}
\begin{document}

\pagestyle{empty}
\begin{picture}(580,845)
%
% Seuraaviin käskyihin täydennetään lomakkeen kenttiin tulevat tiedot.
%
% Tiedekunta/Osasto
%
%%ESIM: \put(58,784){\makebox(100,8)[l]{Matemaattis-luonnontieteellinen}}
\put(58,784){\makebox(100,8)[l]{Matemaattis-luonnontieteellinen }}
%
% Laitos
%
%%ESIM: \put(289,784){\makebox(100,8)[l]{Tietojenkäsittelytieteen laitos}}
\put(289,784){\makebox(100,8)[l]{Matematiikan ja tilastotieteen osasto }}
%
% Tekijä    Tekijän nimi, etunimistä puhuttelunimenä käytetty
%
\put(58,761){\makebox(100,8)[l]{Pekka Keipi}}
%
% Työn nimi
%
\put(58,727){\makebox(200,8)[l]{Uniformiset avaruudet ja Stone–Čech kompaktisointi}}
%
% Oppiaine
%
%%ESIM: \put(58,704){\makebox(100,8)[l]{Tietojenkäsittelyoppi}}
\put(58,704){\makebox(100,8)[l]{Matematiikka }}
%
% Työn laji     pro gradu -, laudatur- tai lisensiaatintyö
%
\put(58,681){\makebox(100,8)[l]{Pro gradu -tutkielma}}
%
% Aika      Työn jättökuukausi ja -vuosi
%
\put(212,681){\makebox(100,8)[l]{Kesäkuu 2018}}
%
% Sivumäärä
%
\put(366,681){\makebox(100,8)[l]{33 s.}}
%
% Avainsanoja
%
\put(58,94){\makebox(100,8)[l]{Topologia, uniforminen avaruus, uniformiteetti, täysin säännöllinen avaruus, kompaktisointi
}}
%
% Säilytyspaikka, sarjanumero
%
\put(58,72){\makebox(100,8)[l]{Kumpulan tiedekirjasto}}
%
%Muita Tietoja
%
\put(58,35){\makebox(100,8)[l]{ }}
%
% Tiivistelmäteksti
%
\put(58,650){\parbox[t]{5.95in}{

%\noindent

Tutkielman tavoitteena on esitellä ja konstruoida Stone–Čech kompaktisointi täysin säännöllisille avaruuksille. Tutkielmassa esitellään myös uniforminen avaruus ja käsitellään tämän yhteyttä pseudometriikoihin. 

Tutkielman alussa käydään läpi käytettäviä topologiaan ja joukkomerkintöihin liittyviä käsitteitä ja merkintätapoja. 
Lukijan oletetaan tuntevan yleisen topologisen avaruuden määritelmä ja tähän liittyviä perustuloksia. 

%Täysin säännöllinen avaruus määritellään uniformisen avaruuden avulla. 
Peruskäsitteiden jälkeen esitellään uniforminen avaruus, eli 
%Uniforminen avaruus on 
topologinen avaruus, johon on lisätty uniforminen rakenne. 
Tämä rakenne mahdollistaa muun muassa täydellisyyden ja tasaisen jatkuvuuden määrittelyn ilman metriikkaa. 
Hausdorff uniformisoituvalle avaruudelle, eli täysin säännölliselle avaruudelle voidaan konstruoida Stone–Čech kompaktisointi.

%Stone–Čech kompaktisointi voidaan konstruoida useilla keskenään yhtäpitävillä tavoilla, muun muassa ultrafilttereillä tai $C^*$-algebroilla. 
Tässä tutkielmassa Stone–Čech kompaktisointi konstruoidaan käyttäen upotusta yksikkövälien tuloon. 
Tulos voitaisiin konstruoida yhtäpitävästi myös muilla tavoilla, muun muassa ultrafilttereillä tai $C^*$-algebroilla. 
Näitä muita tapoja emme kuitenkaan tässä työssä käsittele. 

}}

%%Tarvittaessa tekstiä tiivistelmälaatikon alareunaan:
%%Esimerkiksi luokitustiedot tms.
\put(58,210){\parbox[t]{5.95in}{

}}
%%%%%%%%Seuraaviin kohtiin ei  tarvitse tehdä muutoksia
%
% iso kehys
%
\put(53,30){\framebox(462,786){}}
%
% vaakaviivat
%
\put(53,779){\line(1,0){462}}
\put(53,756.24){\line(1,0){462}}
\put(53,722.1){\line(1,0){462}}
\put(53,699.34){\line(1,0){462}}
\put(53,676.58){\line(1,0){462}}
\put(53,67){\line(1,0){462}}
\put(53,89.76){\line(1,0){462}}
\put(53,112.52){\line(1,0){462}}
%
% pystyviivat
%
\put(284,779){\line(0,1){37}}
\put(207,676.58){\line(0,1){22.76}}
\put(361,676.58){\line(0,1){22.76}}

\put(58,809){\makebox(150,6)[l]{
\tiny Tiedekunta\division Osasto --- Fakultet\division Sektion --- Faculty}}
\put(289,809){\makebox(100,6)[l]{\tiny Laitos --- Institution --- Department}}
\put(58,773){\makebox(100,5)[l]{\tiny Tekij\"a --- F\"orfattare --- Author}}
\put(58,750){\makebox(100,5)[l]{\tiny Ty\"on nimi --- Arbetets titel --- Title}}
\put(58,716){\makebox(100,5)[l]{\tiny Oppiaine --- L\"aro\"amne --- Subject}}
\put(58,693){\makebox(100,5)[l]{\tiny Ty\"on laji --- Arbetets art --- Level}}
\put(212,693){\makebox(100,5)[l]{\tiny Aika --- Datum --- Month and year }}
\put(366,693){\makebox(100,5)[l]{\tiny Sivum\"a\"ar\"a --- Sidoantal --- 
    Number of pages}}
\put(58,670){\makebox(100,5)[l]{\tiny Tiivistelm\"a --- Referat --- Abstract}}
\put(58,106){\makebox(100,5)[l]{\tiny Avainsanat --- Nyckelord --- Keywords}}
\put(58,83){\makebox(100,5)[l]{\tiny S\"ailytyspaikka --- F\"orvaringsst\"alle
--- Where deposited}}
\put(58,61){\makebox(100,5)[l]{\tiny Muita tietoja --- \"Ovriga uppgifter
--- Additional information}}
\put(53,821){\makebox(100,8)[l]{HELSINGIN YLIOPISTO --- HELSINGFORS
UNIVERSITET --- UNIVERSITY OF HELSINKI}}

\end{picture}
\end{document}

